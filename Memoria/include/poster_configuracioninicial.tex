%%%%%%%%%%%%%%%%%%%%%%%%%%%%%%%%%%%%%%%%%%%%%%%%%%%%%%%%%%%%%%%%%%%%%%%%
% Plantilla Póster TFG/TFM
% Escuela Politécnica Superior de la Universidad de Alicante
% Realizado por: Jose Manuel Requena Plens
% Contacto: info@jmrplens.com / Telegram:@jmrplens
%%%%%%%%%%%%%%%%%%%%%%%%%%%%%%%%%%%%%%%%%%%%%%%%%%%%%%%%%%%%%%%%%%%%%%%%

% ESTE ARCHIVO CONTIENE TANTO FORMATO Y PAQUETES PARA EL PÓSTER COMO
% CONFIGURACIÓN DE CABECERA, PIE DE PÁGINA Y ESTILOS

%%%%%%%%%%%%%%%%%%%%%%%%
% FORMATO DEL DOCUMENTO
%%%%%%%%%%%%%%%%%%%%%%%%
% 'tikzposter' es la clase de documento
% Esta configurado para papel DINA0. Ver el manual de tikzposter para más información
\documentclass[
		25pt,a0paper,portrait,
		margin				= 0mm, 	% Margen general (no modificar)
		innermargin			= 12mm,	% Margen interno
		blockverticalspace	= 15mm,	% Espacio vertical entre bloques
		colspace			= 12mm,	% Espacio horizontal entre columnas
		subcolspace			= 8mm	% Espacio horizontal entre subcolumnas
		]{tikzposter}  

% Para modificar el tamaño establecido del poster (DINA0) incluir a continuación
% las dimensiones
\geometry{paperwidth=70cm,paperheight=100cm}


% Gestión de los márgenes para cualquier tamaño de página
\makeatletter
\setlength{\TP@visibletextwidth}{\textwidth-2\TP@innermargin}
\setlength{\TP@visibletextheight}{\textheight-2\TP@innermargin}
\makeatother

% ************************************************
%
%	PAQUETES Y COMANDOS
%
% ************************************************
    
% Elimina los 'warnings' de problemas conocidos. 
\usepackage{silence}
\WarningFilter{latexfont}{Font shape}
\WarningFilter{latexfont}{Some font}

% Fuentes de la plantilla
\usepackage{mathpazo}
\usepackage{fontspec}
\setmainfont{Kurier}[
	% HyphenChar=None, % No dividir palabras al saltar de línea
	Path		= ./include/fuentes/,
	Extension 	= .otf,
	UprightFont = *Light-Regular,
	BoldFont 	= *-Regular,
	ItalicFont 	= *Light-Italic,
	BoldItalicFont = *-Italic
	]


%%%%%%%%%%%%%%%%%%%%%%%%
% BIBLIOGRAFÍA
%%%%%%%%%%%%%%%%%%%%%%%%
\usepackage{apacite} % NORMA APA
\usepackage{natbib}
\usepackage{breakcites}

%%%%%%%%%%%%%%%%%%%%%%%%
% DOCUMENTO EN ESPAÑOL
%%%%%%%%%%%%%%%%%%%%%%%%
\usepackage[base]{babel}
\usepackage{polyglossia}
\setdefaultlanguage{spanish}

%%%%%%%%%%%%%%%%%%%%%%%%
% TABLAS
%%%%%%%%%%%%%%%%%%%%%%%%
% Paquetes para tablas
\usepackage{longtable,booktabs,array,multirow,tabularx,ragged2e,array,rotating}
% Nuevos tipos de columna para tabla, se pueden utilizar como por ejemplo C{3cm} en la definición de columnas de la función tabular
\newcolumntype{L}[1]{>{\raggedright\let\newline\\\arraybackslash\hspace{0pt}}m{#1}}
\newcolumntype{C}[1]{>{\centering\let\newline\\\arraybackslash\hspace{0pt}}m{#1}}
\newcolumntype{R}[1]{>{\raggedleft\let\newline\\\arraybackslash\hspace{0pt}}m{#1}}

%%%%%%%%%%%%%%%%%%%%%%%% 
% FIGURAS, TABLAS, ETC 
%%%%%%%%%%%%%%%%%%%%%%%% 
\usepackage{subcaption} % Para poder realizar subfiguras
\usepackage{caption} % Para aumentar las opciones de diseño
% Nombres de figuras, tablas, etc, en negrita la numeración, todo con letra small
\captionsetup{labelfont={bf,small},textfont=small}
% Paquete para modificar los espacios arriba y abajo de una figura o tabla
\usepackage{setspace}
% Define el espacio tanto arriba como abajo de las figuras, tablas
\setlength{\intextsep}{5mm}
% Para ajustar tamaños de texto de toda una tabla o grafica
% Uso: {\scalefont{0.8} \begin{...} \end{...} }
\usepackage{scalefnt}

%%%%%%%%%%%%%%%%%%%%%%%% 
% GRAFICOS 
%%%%%%%%%%%%%%%%%%%%%%%%
\usetikzlibrary{decorations.text,spy,mindmap, trees, shadows,shadings}

%%%%%%%%%%%%%%%%%%%%%%%% 
% TEXTO
%%%%%%%%%%%%%%%%%%%%%%%%
% Paquete para poder modificar las fuente de texto
%\usepackage{xltxtra}
% Cualquier tamaño de texto. Uso: {\fontsize{100pt}{120pt}\selectfont tutexto}
\usepackage{anyfontsize}
% Para modificar parametros del texto.
\usepackage{setspace}

%%%%%%%%%%%%%%%%%%%%%%%% 
% GLOSARIOS
%%%%%%%%%%%%%%%%%%%%%%%%
\usepackage[acronym]{glossaries}

%%%%%%%%%%%%%%%%%%%%%%%% 
% MATEMÁTICAS
%%%%%%%%%%%%%%%%%%%%%%%%
\usepackage{amsmath,latexsym,mathtools,amsthm,amsfonts,amssymb,bm,mathrsfs} 
\usepackage{upgreek}
% Comando para añadir información de variables a las ecuaciones
% Uso: \begin{condiciones}[donde:] ....... \end{condiciones}
\newenvironment{condiciones}[1][2]
  {%
   #1\tabularx{\textwidth-\widthof{#1}}[t]{
     >{$}l<{$} @{}>{${}}c<{{}$}@{} >{\raggedright\arraybackslash}X
   }%
  }
  {\endtabularx\\[\belowdisplayskip]}

%%%%%%%%%%%%%%%%%%%%%%%% 
% OTROS
%%%%%%%%%%%%%%%%%%%%%%%%
% Para introducir url's con formato. Uso: \url{http://www.google.es}
\usepackage{url}
% Amplia muchas funciones gráficas de latex
\usepackage{graphicx}
% Paquete que añade el hipervinculo en referencias dentro del documento, indice, etc
% Se define sin bordes alrededor. Uso: \ref{tulabel}
\usepackage[pdfborder={000}]{hyperref}
\usepackage{verbatim}
% Paquete para condicionales avanzados
\usepackage{xstring,xifthen}
% Paquete para realizar cálculos en el código
\usepackage{calc}
% Para incluir comentarios en el texto. El parámetro 'disable' oculta todas las notas.
% USO: \todo{tutexto}
\usepackage[textsize=tiny,spanish,shadow,textwidth=2cm]{todonotes}
% Comando para generar un nodo a un bloque. Le pone nombre a un bloque para luego poder
% generar gráficos que apunten a ese bloque o a varios bloques. Se escribe justo debajo
% del comando \block.
% USO: \labelblock{BloqueA}  
\newcommand\labelblock[1]{\node[fit=(blockbody)(blocktitle),inner sep=5pt] (#1) {};}
% Abreviatura de la barra invertida
\newcommand{\bs}{\textbackslash}   


% ************************************************
%
%	CONFIGURACIONES DE ESTILOS Y COLORES
%
% ************************************************

% Obtiene los colores de la titulación y logotipos según la ID elegida en el archivo principal 'poster_TFG-TFM_EPS_UA.tex'
%%%%%%%%%%%%%%%%%%%%%%%%%%%%%%%%%%%%%%%%%%%%%%%%%%%%%%%%%%%%%%%%%%%%%%%%
% Plantilla TFG/TFM
% Escuela Politécnica Superior de la Universidad de Alicante
% Realizado por: Jose Manuel Requena Plens
% Contacto: info@jmrplens.com / Telegram:@jmrplens
%%%%%%%%%%%%%%%%%%%%%%%%%%%%%%%%%%%%%%%%%%%%%%%%%%%%%%%%%%%%%%%%%%%%%%%%

%%%%%%%%%%%%%%%%%%%%%%%% 
% COLORES DE GRADOS.
% Si el color de la titulación ha cambiado, modifícalo en las lineas siguientes.
%%%%%%%%%%%%%%%%%%%%%%%%
% Grados
\definecolor{teleco}{RGB}{32,2,116}			% Teleco
\definecolor{civil}{RGB}{201,56,140}			% Civil
\definecolor{quimica}{RGB}{41,199,255}		% Química
\definecolor{informatica}{RGB}{0,128,255}	% Informatica
\definecolor{multimedia}{RGB}{239,206,53}	% Multimedia
\definecolor{arquitecnica}{RGB}{0,179,148}	% Arquitectura técnica
\definecolor{arquitectura}{RGB}{181,0,0}		% Arquitectura
\definecolor{robotica}{RGB}{255,255,128}		% Robótica
% Másteres
\definecolor{masterteleco}{RGB}{32,2,116}	% Teleco
\definecolor{caminos}{RGB}{201,56,140}		% Caminos, Canales y Puertos
\definecolor{gestedif}{RGB}{50,120,50}		% Gestión Edificación
\definecolor{desweb}{RGB}{250,43,22}			% Desarrollo Web
\definecolor{mataguaterre}{RGB}{210,250,50}	% Materiales, Agua, Terreno
\definecolor{masterinfor}{RGB}{0,128,255}	% Informática
\definecolor{autorobo}{RGB}{83,145,201}		% Automática y Robótica
\definecolor{prevencion}{RGB}{0,100,0}		% Prevención Riesgos
\definecolor{gestionagua}{RGB}{7,138,197}	% Gestión Agua
\definecolor{moviles}{RGB}{121,11,21}		% Aplicaciones Móviles
\definecolor{masterquimica}{RGB}{41,199,255}	% Quimica
\definecolor{ciberseguridad}{RGB}{9,111,192}	% Ciberseguridad

% Logotipos comunes de todas las titulaciones
\newcommand{\logoFacultad}{include/logos-universidad/LogoEPSNegro}
\newcommand{\logoUniversidad}{include/logos-universidad/LogoUANegro}
\newcommand{\logoUniversidadPortada}{include/logos-universidad/LogoUABlanco}

% Colores generales
\definecolor{negro}{RGB}{0,0,0}
\definecolor{blanco}{RGB}{255,255,255}
%%%%%%%%%%%%%%%%%%%%%%%% 
% CONDICIONALES. SEGUN LA ID ELEGIDA EN EL .TEX PRINCIPAL
% Según el ID seleccionado en TFG_EPS_UA.tex se configurará el nombre de la titulación, logotipos y color.
% Si tu titulación no esta correctamente definida cambia las imágenes que se definen para tu titulación en las lineas de abajo
% Si deseas añadir mas titulaciones ve al final de este archivo
%%%%%%%%%%%%%%%%%%%%%%%%
% Grados
	\if\IDtitulo 1 % Teleco
		% Logos
		\newcommand{\logoFacultadPortada}{include/logos-universidad/LogoEPSBlanco}
		\newcommand{\logoGradoPortada}{include/logos-titulaciones/LogoTelecoBlanco}
		\newcommand{\logoGrado}{include/logos-titulaciones/LogoTelecoNegro}
		% Texto
		\newcommand{\miGrado}{Grado en Ingeniería en Sonido e Imagen en Telecomunicación}
		\newcommand{\tipotrabajo}{Trabajo Fin de Grado}
		% Color
		\newcommand{\colorgrado}{teleco}
		\newcommand{\colortexto}{blanco}
	\else \if\IDtitulo 2 % Civil
		\newcommand{\logoFacultadPortada}{include/logos-universidad/LogoEPSBlanco}
		\newcommand{\logoGradoPortada}{include/logos-titulaciones/LogoCivilBlanco}
		\newcommand{\logoGrado}{include/logos-titulaciones/LogoCivilNegro}
		% Texto
		\newcommand{\miGrado}{Grado en Ingeniería Civil}
		\newcommand{\tipotrabajo}{Trabajo Fin de Grado}
		% Color
		\newcommand{\colorgrado}{civil}
		\newcommand{\colortexto}{blanco}
	\else \if\IDtitulo 3 % Quimica
		% Logos
		\newcommand{\logoFacultadPortada}{include/logos-universidad/LogoEPSNegro}
		\newcommand{\logoGradoPortada}{include/logos-titulaciones/LogoQuimicaNegro}
		\newcommand{\logoGrado}{include/logos-titulaciones/LogoQuimicaNegro}
		% Texto
		\newcommand{\miGrado}{Grado en Ingeniería Química}
		\newcommand{\tipotrabajo}{Trabajo Fin de Grado}
		% Color
		\newcommand{\colorgrado}{quimica}
		\newcommand{\colortexto}{negro}
	\else \if\IDtitulo 4 % Informatica
		% Logos
		\newcommand{\logoFacultadPortada}{include/logos-universidad/LogoEPSBlanco}
		\newcommand{\logoGradoPortada}{include/logos-titulaciones/LogoInformaticaBlanco}
		\newcommand{\logoGrado}{include/logos-titulaciones/LogoInformaticaNegro}
		% Texto
		\newcommand{\miGrado}{Grado en Ingeniería Informática}
		\newcommand{\tipotrabajo}{Trabajo Fin de Grado}
		% Color
		\newcommand{\colorgrado}{informatica}
		\newcommand{\colortexto}{blanco}
	\else \if\IDtitulo 5 % Multimedia
		% Logos
		\newcommand{\logoFacultadPortada}{include/logos-universidad/LogoEPSNegro}
		\newcommand{\logoGradoPortada}{include/logos-titulaciones/LogoMultimediaNegro}
		\newcommand{\logoGrado}{include/logos-titulaciones/LogoMultimediaNegro}
		% Texto
		\newcommand{\miGrado}{Grado en Ingeniería Multimedia}
		\newcommand{\tipotrabajo}{Trabajo Fin de Grado}
		% Color
		\newcommand{\colorgrado}{multimedia}
		\newcommand{\colortexto}{negro}
	\else \if\IDtitulo 6 % Arquitectura Tecnica
		% Logos
		\newcommand{\logoFacultadPortada}{include/logos-universidad/LogoEPSBlanco}
		\newcommand{\logoGradoPortada}{include/logos-titulaciones/LogoArqTecnicaBlanco}
		\newcommand{\logoGrado}{include/logos-titulaciones/LogoArqTecnicaNegro}
		% Texto
		\newcommand{\miGrado}{Grado en Arquitectura Técnica}
		\newcommand{\tipotrabajo}{Trabajo Fin de Grado}
		% Color
		\newcommand{\colorgrado}{arquitecnica}
		\newcommand{\colortexto}{blanco}
	\else \if\IDtitulo 7 % Arquitectura
		% Logos
		\newcommand{\logoFacultadPortada}{include/logos-universidad/LogoEPSBlanco}
		\newcommand{\logoGradoPortada}{include/logos-titulaciones/LogoArquitecturaBlanco}
		\newcommand{\logoGrado}{include/logos-titulaciones/LogoArquitecturaNegro}
		% Texto
		\newcommand{\miGrado}{Grado en Arquitectura}
		\newcommand{\tipotrabajo}{Trabajo Fin de Grado}
		% Color
		\newcommand{\colorgrado}{arquitectura}
		\newcommand{\colortexto}{blanco}
	\else \if\IDtitulo 8 % Robotica
		% Logos
		\newcommand{\logoFacultadPortada}{include/logos-universidad/LogoEPSNegro}
		\newcommand{\logoGradoPortada}{include/logos-titulaciones/LogoRoboticaColor}
		\newcommand{\logoGrado}{include/logos-titulaciones/LogoRoboticaNegro}
		% Texto
		\newcommand{\miGrado}{Grado en Ingeniería Robótica}
		\newcommand{\tipotrabajo}{Trabajo Fin de Grado}
		% Color
		\newcommand{\colorgrado}{robotica}
		\newcommand{\colortexto}{negro}
% Másteres
	\else \if\IDtitulo A % Teleco
		% Logos
		\newcommand{\logoFacultadPortada}{include/logos-universidad/LogoEPSBlanco}
		\newcommand{\logoGradoPortada}{include/logos-titulaciones/LogoTelecoBlanco}
		\newcommand{\logoGrado}{include/logos-titulaciones/LogoTelecoNegro}
		% Texto
		\newcommand{\miGrado}{Máster Universitario en Ingeniería en Telecomunicación}
		\newcommand{\tipotrabajo}{Trabajo Fin de Máster}
		% Color
		\newcommand{\colorgrado}{masterteleco}
		\newcommand{\colortexto}{blanco}
	\else \if\IDtitulo B % Caminos, Canales y puertos
		% Logos
		\newcommand{\logoFacultadPortada}{include/logos-universidad/LogoEPSBlanco}
		\newcommand{\logoGradoPortada}{include/logos-titulaciones/LogoCivilBlanco}
		\newcommand{\logoGrado}{include/logos-titulaciones/LogoCivilNegro}
		% Texto
		\newcommand{\miGrado}{Máster Universitario en Ingeniería de Caminos, Canales y Puertos}
		\newcommand{\tipotrabajo}{Trabajo Fin de Máster}
		% Color
		\newcommand{\colorgrado}{caminos}
		\newcommand{\colortexto}{blanco}
	\else \if\IDtitulo C % Gestión Edificación
		% Logos
		\newcommand{\logoFacultadPortada}{include/logos-universidad/LogoEPSBlanco}
		\newcommand{\logoGradoPortada}{include/logos-titulaciones/LogoMasterEdificacionBlanco}
		\newcommand{\logoGrado}{include/logos-titulaciones/LogoMasterEdificacionNegro}
		\newcommand{\tipotrabajo}{Trabajo Fin de Máster}
		% Texto
		\newcommand{\miGrado}{Máster Universitario en Gestión de la Edificación}
		% Color
		\newcommand{\colorgrado}{gestedif}
		\newcommand{\colortexto}{blanco}
	\else \if\IDtitulo D % Desarrollo web
		% Logos
		\newcommand{\logoFacultadPortada}{include/logos-universidad/LogoEPSBlanco}
		\newcommand{\logoGradoPortada}{include/logos-titulaciones/LogoMasterDesarrolloBlanco}
		\newcommand{\logoGrado}{include/logos-titulaciones/LogoMasterDesarrolloNegro}
		% Texto
		\newcommand{\miGrado}{Máster Universitario en Desarrollo de Aplicaciones y Servicios Web}
		\newcommand{\tipotrabajo}{Trabajo Fin de Máster}
		% Color
		\newcommand{\colorgrado}{desweb}
		\newcommand{\colortexto}{blanco}
	\else \if\IDtitulo E % Materiales, Agua, Terreno
		% Logos
		\newcommand{\logoFacultadPortada}{include/logos-universidad/LogoEPSNegro}
		\newcommand{\logoGradoPortada}{include/logos-titulaciones/LogoMasterMaterialesNegro}
		\newcommand{\logoGrado}{include/logos-titulaciones/LogoMasterMaterialesNegro}
		% Texto
		\newcommand{\miGrado}{Máster Universitario en Ingeniería de los Materiales, del Agua y del Terreno}
		\newcommand{\tipotrabajo}{Trabajo Fin de Máster}
		% Color
		\newcommand{\colorgrado}{mataguaterre}
		\newcommand{\colortexto}{negro}
	\else \if\IDtitulo F % Informatica
		% Logos
		\newcommand{\logoFacultadPortada}{include/logos-universidad/LogoEPSBlanco}
		\newcommand{\logoGradoPortada}{include/logos-titulaciones/LogoInformaticaBlanco}
		\newcommand{\logoGrado}{include/logos-titulaciones/LogoInformaticaNegro}
		% Texto
		\newcommand{\miGrado}{Máster Universitario en Ingeniería Informática}
		\newcommand{\tipotrabajo}{Trabajo Fin de Máster}
		% Color
		\newcommand{\colorgrado}{masterinfor}
		\newcommand{\colortexto}{blanco}
	\else \if\IDtitulo G % Automática y Robótica
		% Logos
		\newcommand{\logoFacultadPortada}{include/logos-universidad/LogoEPSBlanco}
		\newcommand{\logoGradoPortada}{include/logos-titulaciones/LogoMasterRoboticaBlanco}
		\newcommand{\logoGrado}{include/logos-titulaciones/LogoMasterRoboticaNegro}
		% Texto
		\newcommand{\miGrado}{Máster Universitario en Automática y Robótica}
		\newcommand{\tipotrabajo}{Trabajo Fin de Máster}
		% Color
		\newcommand{\colorgrado}{autorobo}
		\newcommand{\colortexto}{blanco}
	\else \if\IDtitulo H % Prevención de riesgos laborales
		% Logos
		\newcommand{\logoFacultadPortada}{include/logos-universidad/LogoEPSBlanco}
		\newcommand{\logoGradoPortada}{include/logos-titulaciones/LogoMasterPrevencionBlanco}
		\newcommand{\logoGrado}{include/logos-titulaciones/LogoMasterPrevencionNegro}
		% Texto
		\newcommand{\miGrado}{Máster Universitario en Prevención de Riesgos Laborales}
		\newcommand{\tipotrabajo}{Trabajo Fin de Máster}
		% Color
		\newcommand{\colorgrado}{prevencion}
		\newcommand{\colortexto}{blanco}
	\else \if\IDtitulo I % Gestion Agua
		% Logos
		\newcommand{\logoFacultadPortada}{include/logos-universidad/LogoEPSNegro}
		\newcommand{\logoGradoPortada}{include/logos-titulaciones/LogoMasterAguaNegro}
		\newcommand{\logoGrado}{include/logos-titulaciones/LogoMasterAguaNegro}
		% Texto
		\newcommand{\miGrado}{Máster Universitario en Gestión Sostenible y Tecnologías del Agua}
		\newcommand{\tipotrabajo}{Trabajo Fin de Máster}
		% Color
		\newcommand{\colorgrado}{gestionagua}
		\newcommand{\colortexto}{negro}
	\else \if\IDtitulo J % Aplicaciones Móviles
		% Logos
		\newcommand{\logoFacultadPortada}{include/logos-universidad/LogoEPSBlanco}
		\newcommand{\logoGradoPortada}{include/logos-titulaciones/LogoMasterMovilesBlanco}
		\newcommand{\logoGrado}{include/logos-titulaciones/LogoMasterMovilesNegro}
		% Texto
		\newcommand{\miGrado}{Máster Universitario en Desarrollo de Software para Dispositivos Móviles}
		\newcommand{\tipotrabajo}{Trabajo Fin de Máster}
		% Color
		\newcommand{\colorgrado}{moviles}
		\newcommand{\colortexto}{blanco}
	\else \if\IDtitulo K % Quimica
		% Logos
		\newcommand{\logoFacultadPortada}{include/logos-universidad/LogoEPSNegro}
		\newcommand{\logoGradoPortada}{include/logos-titulaciones/LogoQuimicaNegro}
		\newcommand{\logoGrado}{include/logos-titulaciones/LogoQuimicaNegro}
		% Texto
		\newcommand{\miGrado}{Máster Universitario en Ingeniería Química}
		\newcommand{\tipotrabajo}{Trabajo Fin de Máster}
		% Color
		\newcommand{\colorgrado}{masterquimica}
		\newcommand{\colortexto}{negro}
	\else \if\IDtitulo L % Ciberseguridad
		% Logos
		\newcommand{\logoFacultadPortada}{include/logos-universidad/LogoEPSBlanco}
		\newcommand{\logoGradoPortada}{include/logos-titulaciones/LogoMasterCiberseguridadColor}
		\newcommand{\logoGrado}{include/logos-titulaciones/LogoMasterCiberseguridadNegro}
		% Texto
		\newcommand{\miGrado}{Máster Universitario en Ciberseguridad}
		\newcommand{\tipotrabajo}{Trabajo Fin de Máster}
		% Color
		\newcommand{\colorgrado}{ciberseguridad}
		\newcommand{\colortexto}{blanco}


	\fi \fi \fi \fi \fi \fi \fi \fi \fi \fi \fi \fi \fi \fi \fi \fi \fi \fi \fi \fi
	
%%%%%%%%%%%%%%%%%%%%%%%%%%%%%%%%%%%%%%%%%%%%%%%%%%%%%%%%%%%%%%%%%%%%%%%%	
% ¿COMO AÑADIR MÁS TITULACIONES?
% Para añadir más titulaciones, se debe continuar el el formato de ID -> Titulacion.
% Justo encima de la linea donde hay muchos '\fi' se debe escribir el condicional y el contenido de este tal que:
%
%	\else \if\IDtitulo X % Titulacion con ID=X		
% 		% Logos
%		\newcommand{\logoFacultadPortada}{include/logos-universidad/LogoEPSBlanco}
%		\newcommand{\logoGradoPortada}{include/logos-titulaciones/logotitulacion}
%		\newcommand{\logoGrado}{include/logos-titulaciones/logotitulacion}
%		% Texto
%		\newcommand{\miGrado}{Grado en XXXXXXXX}
%		\newcommand{\tipotrabajo}{Trabajo Fin de XXXX}
%		% Color
%		\newcommand{\colorgrado}{XXXX}
%		\newcommand{\colortexto}{XXX}
%	
% Por último añadir a la linea que tiene muchos '\fi', otro '\fi'. Listo, ya podrás usar la nueva ID con la configuración añadida.
%%%%%%%%%%%%%%%%%%%%%%%%%%%%%%%%%%%%%%%%%%%%%%%%%%%%%%%%%%%%%%%%%%%%%%%%	






%%%%%%%%%%%%%%%%%%%%%%%% 
% CABECERA Y PIE DE PÁGINA. NO MODIFICAR.
%%%%%%%%%%%%%%%%%%%%%%%%
% Tema de la plantilla. (NO MODIFICAR)
\usetheme{Default}
% Color título (NO MODIFICAR)
\colorlet{titlefgcolor}{\colortexto}	
\colorlet{titlebgcolor}{\colorgrado}
% Define el estilo de la cabecera. (NO MODIFICAR)
\definetitlestyle{TFGTFM}{
    width=\paperwidth, roundedcorners=0, linewidth=0pt, innersep=1.5cm,
    titletotopverticalspace=0mm, titletoblockverticalspace=20mm,
    titlegraphictotitledistance=10pt
}{
   \draw[draw=none, fill=titlebgcolor]%
   (\titleposleft,\titleposbottom) rectangle (\titleposright,\titlepostop);}
% Establece el estilo de la cabecera. (NO MODIFICAR)
\usetitlestyle{TFGTFM}
% Genera la barra superior
%%%%%%%%%%%%%%%%%%%%%%%%%%%%%%%%%%%%%%%%%%%%%%%%%%%%%%%%%%%%%%%%%%%%%%%%
% Plantilla Póster TFG/TFM
% Escuela Politécnica Superior de la Universidad de Alicante
% Realizado por: Jose Manuel Requena Plens
% Contacto: info@jmrplens.com / Telegram:@jmrplens
%%%%%%%%%%%%%%%%%%%%%%%%%%%%%%%%%%%%%%%%%%%%%%%%%%%%%%%%%%%%%%%%%%%%%%%%

% Establece la fuentes de texto para el titulo y grado
% Helvetica. Uso: {\FuentePortada tutexto}
\newfontfamily\FuentePortada{Helvetica}[Path=./include/fuentes/]  

% Tamaño por defecto de la fuente de texto para:
\def\FuenteTamano{\huge}	% Tamaño para el título del trabajo


% Según la longitud del título se determina un tamaño para él
\StrLen{\titulo}[\longitudtitulo] % Cuenta los caracteres título
% Comprueba la longitud del título y según sea este determina unos valores nuevos
\ifthenelse{\longitudtitulo > 220}{
\def\FuenteTamano{\normalsize}}
{\ifthenelse{\longitudtitulo > 180}{
\def\FuenteTamano{\large}}
{\ifthenelse{\longitudtitulo > 140}{
\def\FuenteTamano{\Large}}	
{\ifthenelse{\longitudtitulo > 120}{
\def\FuenteTamano{\LARGE}} 	
{} % Si no, no modifica el tamaño
} } }

\def\logoFooter{include/logos-universidad/LogoUAColor}

%%%%%%%%%%%%%%%%%%%%%%%%%%%%%%%%%%%%%%%%%%%%
% Barra superior
%%%%%%%%%%%%%%%%%%%%%%%%%%%%%%%%%%%%%%%%%%%%
\def\tabularxcolumn#1{m{#1}}

\settitle{
\vspace{-1cm}
\begin{tabularx}{\linewidth}{Xm{0.06\linewidth}m{0.01\linewidth}m{0.25\linewidth}m{0.01\linewidth}m{0.13\linewidth}}

\centering \color{titlefgcolor} {\FuentePortada\FuenteTamano\titulo}
& 
\raggedleft\includegraphics[height=0.04\paperheight]{\logoGradoPortada} 
& ~ &
\centering \color{titlefgcolor} {\FuentePortada\large\miGrado}
& ~ &
\raggedright\includegraphics[height=0.035\paperheight]{\logoFacultadPortada}
\end{tabularx}
\vspace{-1cm}
}


%%%%%%%%%%%%%%%%%%%%%%%%%%%%%%%%%%%%%%%%%%%%
% Barra inferior
%%%%%%%%%%%%%%%%%%%%%%%%%%%%%%%%%%%%%%%%%%%%
% Pie de página
\def\footer{
	% Caja
	\node [above right,
       outer sep=0pt,
       minimum width=\paperwidth-2*\pgflinewidth,
       minimum height=6cm,
       draw,fill=white] at (bottomleft) {};
    % Información autor   
 	\node [above right,
       outer sep=0pt] at ([shift={(0.5cm,3.5cm)}]bottomleft) {
       \FuentePortada\Large \color{black} Autor/a: \autor
       };
 	% Tipo de trabajo
 	\node [above right,
       outer sep=0pt,minimum height=6cm,minimum width=\paperwidth-2*\pgflinewidth,
       align=center] at ([shift={(0.5*\pgflinewidth,0.5*\pgflinewidth)}]bottomleft) {\LARGE \FuentePortada\color{black}\tipotrabajo};
    % Fecha   
  	\node [above right,
       outer sep=0pt] at ([shift={(0.5cm,1cm)}]bottomleft) {
       \FuentePortada\Large \color{black} \Hoy
       };
  	% Logotipo universidad
  	\node[above left,outer sep=1.5cm] at (bottomleft -| topright)
  	{\includegraphics[height=0.031\paperheight]{\logoFooter}};
       }


%%%%%%%%%%%%%%%%%%%%%%%% 
% ESTILOS, COLORES Y ELEMENTOS GRÁFICOS
%%%%%%%%%%%%%%%%%%%%%%%%
%%%%%%%%%%%%%%%%%%%%%%%%%%%%%%%%%%%%%%%%%%%%%%%%%%%%%%%%%%%%%%%%%%%%%%%%
% Plantilla Póster TFG/TFM
% Escuela Politécnica Superior de la Universidad de Alicante
% Realizado por: Jose Manuel Requena Plens
% Contacto: info@jmrplens.com / Telegram:@jmrplens
%%%%%%%%%%%%%%%%%%%%%%%%%%%%%%%%%%%%%%%%%%%%%%%%%%%%%%%%%%%%%%%%%%%%%%%%

% EN ESTE ARCHIVO SE DEFINEN LOS CONJUNTOS DE COLORES, ESTILOS DE BLOQUES 
% Y ELEMENTOS GRAFICOS COMO FLECHAS, FORMAS O MODIFICACIONES DE BLOQUES

% ÍNDICE
% 1. Conjunto de colores 
% 2. Estilo de bloques
% 3. Estilo de bloques internos
% 4. Estilo de fondos
% 5. Estilo de notas
% 6. Elementos gráficos
%  6.1. Modificadores de bloques
%  6.2. Flechas



%%%%%%%%%%%%%%%%%%%%%%%% 
% 1. CONJUNTO DE COLORES
%%%%%%%%%%%%%%%%%%%%%%%%
% Estilos disponibles de colores
% 'TFGTFM','Default','Australia','Britain','Sweden','Spain','Russia','Denmark','Germany'
\definecolorstyle{TFGTFM} {}{
	% Colores de fondo
	\colorlet{backgroundcolor}{\colorgrado!20!white} % Fondo general
	\colorlet{framecolor}{black}						 % Color de marco (no utilizado)
	% Colores bloques
	\colorlet{blocktitlebgcolor}{\colorgrado}	% Fondo cabecera
	\colorlet{blocktitlefgcolor}{\colortexto}	% Texto cabecera
	\colorlet{blockbodybgcolor}{white}			% Fondo cuerpo
	\colorlet{blockbodyfgcolor}{black}			% Texto cuerpo
	% Colores de bloques internos
	\colorlet{innerblocktitlebgcolor}{white}				% Borde
	\colorlet{innerblocktitlefgcolor}{black}				% Texto cabecera
	\colorlet{innerblockbodybgcolor}{orange!30!white}	% Fondo
	\colorlet{innerblockbodyfgcolor}{black}				% Texto cuerpo
	% Colores de notas
	\colorlet{notefgcolor}{black}				% Texto
	\colorlet{notebgcolor}{yellow!50!white}		% Fondo
	\colorlet{noteframecolor}{yellow}			% Borde
}

\definecolorstyle{Default}{
    \definecolor{colorOne}{HTML}{DDDDDD}
    \definecolor{colorTwo}{HTML}{0066A8}
    \definecolor{colorThree}{HTML}{FCE565}
}{
    % Colores de fondo
    \colorlet{backgroundcolor}{colorOne}
    \colorlet{framecolor}{colorTwo}
    % Colores bloques
    \colorlet{blocktitlebgcolor}{colorTwo}
    \colorlet{blocktitlefgcolor}{white}
    \colorlet{blockbodybgcolor}{white}
    \colorlet{blockbodyfgcolor}{black}
    % Colores de bloques internos
    \colorlet{innerblocktitlebgcolor}{colorThree}
    \colorlet{innerblocktitlefgcolor}{black}
    \colorlet{innerblockbodybgcolor}{white}
    \colorlet{innerblockbodyfgcolor}{black}
    % Colores de notas
    \colorlet{notefgcolor}{black}
    \colorlet{notebgcolor}{colorThree!70!white}
    \colorlet{notefrcolor}{colorThree}
 }

\definecolorstyle{Australia}{
    \definecolor{colorOne}{HTML}{A2E2C7}
    \definecolor{colorTwo}{HTML}{56555A}
    \definecolor{colorThree}{HTML}{C9AECF}
}{
    % Colores de fondo
    \colorlet{backgroundcolor}{colorOne}
    \colorlet{framecolor}{colorOne!50!colorTwo}
    % Colores bloques
    \colorlet{blocktitlebgcolor}{colorTwo}
    \colorlet{blocktitlefgcolor}{white}
    \colorlet{blockbodybgcolor}{white}
    \colorlet{blockbodyfgcolor}{black}
    % Colores de bloques internos
    \colorlet{innerblocktitlebgcolor}{colorThree}
    \colorlet{innerblocktitlefgcolor}{black}
    \colorlet{innerblockbodybgcolor}{white}
    \colorlet{innerblockbodyfgcolor}{black}
    % Colores de notas
    \colorlet{notefgcolor}{black}
    \colorlet{notebgcolor}{colorThree}
    \colorlet{notefrcolor}{colorThree}
 }

\definecolorstyle{Britain}{
    \definecolor{colorOne}{HTML}{116699}
    \definecolor{colorTwo}{HTML}{CCCCCC}
    \definecolor{colorThree}{HTML}{CC6633}
}{
    % Colores de fondo
    \colorlet{backgroundcolor}{colorOne}
    \colorlet{framecolor}{colorTwo}
    % Colores bloques
    \colorlet{blocktitlebgcolor}{colorTwo}
    \colorlet{blocktitlefgcolor}{colorOne}
    \colorlet{blockbodybgcolor}{white}
    \colorlet{blockbodyfgcolor}{black}
    % Colores de bloques internos
    \colorlet{innerblocktitlebgcolor}{colorThree}
    \colorlet{innerblocktitlefgcolor}{white}
    \colorlet{innerblockbodybgcolor}{white}
    \colorlet{innerblockbodyfgcolor}{black}
    % Colores de notas
    \colorlet{notefgcolor}{black}
    \colorlet{notebgcolor}{colorThree!40!white}
    \colorlet{notefrcolor}{colorThree!60!white}
 }

\definecolorstyle{Sweden}{
    \definecolor{colorOne}{HTML}{116699}
    \definecolor{colorTwo}{HTML}{CCCCCC}
    \definecolor{colorThree}{HTML}{CC6633}
}{
    % Colores de fondo
    \colorlet{backgroundcolor}{colorOne!40!white}
    \colorlet{framecolor}{colorTwo}
    % Colores bloques
    \colorlet{blocktitlebgcolor}{colorTwo!70!black}
    \colorlet{blocktitlefgcolor}{colorOne}
    \colorlet{blockbodybgcolor}{white!90!colorTwo}
    \colorlet{blockbodyfgcolor}{black}
    % Colores de bloques internos
    \colorlet{innerblocktitlebgcolor}{colorThree}
    \colorlet{innerblocktitlefgcolor}{white}
    \colorlet{innerblockbodybgcolor}{white}
    \colorlet{innerblockbodyfgcolor}{black}
    % Colores de notas
    \colorlet{notefgcolor}{black}
    \colorlet{notebgcolor}{colorThree!50!white}
    \colorlet{notefrcolor}{colorThree!50!white}
 }

\definecolorstyle{Spain}{
    \definecolor{colorOne}{HTML}{116699}
    \definecolor{colorTwo}{HTML}{CCCCCC}
    \definecolor{colorThree}{HTML}{CC6633}
}{
    % Colores de fondo
    \colorlet{backgroundcolor}{colorOne!55!white}
    \colorlet{framecolor}{colorTwo}
    % Colores bloques
    \colorlet{blocktitlebgcolor}{colorOne!80!black}
    \colorlet{blocktitlefgcolor}{white}
    \colorlet{blockbodybgcolor}{white}
    \colorlet{blockbodyfgcolor}{black}
    % Colores de bloques internos
    \colorlet{innerblocktitlebgcolor}{colorThree}
    \colorlet{innerblocktitlefgcolor}{white}
    \colorlet{innerblockbodybgcolor}{white}
    \colorlet{innerblockbodyfgcolor}{black}
    % Colores de notas
    \colorlet{notefgcolor}{black}
    \colorlet{notebgcolor}{colorThree!50!white}
    \colorlet{notefrcolor}{colorThree}
 }

\definecolorstyle{Russia}{
    \definecolor{colorOne}{HTML}{116699}
    \definecolor{colorTwo}{HTML}{CCCCCC}
    \definecolor{colorThree}{HTML}{CC6633}
}{
    % Colores de fondo
    \colorlet{backgroundcolor}{white}
    \colorlet{framecolor}{colorOne!50!colorThree!30!}
    % Colores bloques
    \colorlet{blocktitlebgcolor}{colorThree!80!colorTwo!80!black}
    \colorlet{blocktitlefgcolor}{white}
    \colorlet{blockbodybgcolor}{colorTwo!40}
    \colorlet{blockbodyfgcolor}{black}
    % Colores de bloques internos
    \colorlet{innerblocktitlebgcolor}{colorTwo!40}
    \colorlet{innerblocktitlefgcolor}{black}
    \colorlet{innerblockbodybgcolor}{colorTwo}
    \colorlet{innerblockbodyfgcolor}{black}
    % Colores de notas
    \colorlet{notefgcolor}{black}
    \colorlet{notebgcolor}{colorTwo}
    \colorlet{notefrcolor}{colorTwo}
 }
 
\definecolorstyle{Denmark}{
    \definecolor{colorOne}{HTML}{AE0D45}
    \definecolor{colorTwo}{HTML}{7F8897}
    \definecolor{colorThree}{HTML}{C8512D}
}{
    % Colores de fondo
    \colorlet{backgroundcolor}{white}
    \colorlet{framecolor}{white}
    % Colores bloques
    \colorlet{blocktitlebgcolor}{colorTwo}
    \colorlet{blocktitlefgcolor}{colorOne}
    \colorlet{blockbodybgcolor}{white}
    \colorlet{blockbodyfgcolor}{black}
    % Colores de bloques internos
    \colorlet{innerblocktitlebgcolor}{colorThree}
    \colorlet{innerblocktitlefgcolor}{white}
    \colorlet{innerblockbodybgcolor}{white}
    \colorlet{innerblockbodyfgcolor}{black}
    % Colores de notas
    \colorlet{notefgcolor}{black}
    \colorlet{notebgcolor}{colorTwo!50!white}
    \colorlet{notefrcolor}{colorTwo!50!white}
 }

\definecolorstyle{Germany}{
    \definecolor{colorOne}{HTML}{8C7269}
    \definecolor{colorTwo}{HTML}{E89261}
    \definecolor{colorThree}{HTML}{A2C4D9}
}{
    % Colores de fondo
    \colorlet{backgroundcolor}{colorTwo}
    \colorlet{framecolor}{colorThree}
    % Colores bloques
    \colorlet{blocktitlebgcolor}{white}
    \colorlet{blocktitlefgcolor}{colorOne}
    \colorlet{blockbodybgcolor}{white}
    \colorlet{blockbodyfgcolor}{black}
    % Colores de bloques internos
    \colorlet{innerblocktitlebgcolor}{white}
    \colorlet{innerblocktitlefgcolor}{black}
    \colorlet{innerblockbodybgcolor}{colorThree}
    \colorlet{innerblockbodyfgcolor}{black}
    % Colores de notas
    \colorlet{notefgcolor}{black}
    \colorlet{notebgcolor}{colorThree}
    \colorlet{notefrcolor}{colorThree}
 }
 
\definecolorstyle{Qacolor}{
    \definecolor{colorOne}{HTML}{BE0920}
    \definecolor{colorTwo}{HTML}{7F8897}
    \definecolor{colorThree}{HTML}{C8512D}
    \definecolor{colorFour}{HTML}{1E0D85}
}{
    % Colores de fondo
    \colorlet{backgroundcolor}{white}
    \colorlet{framecolor}{white}
    % Colores bloques
    \colorlet{blocktitlebgcolor}{colorTwo!35!white}
    \colorlet{blocktitlefgcolor}{colorOne}
    \colorlet{blockbodybgcolor}{white}
    \colorlet{blockbodyfgcolor}{black}
    % Colores de bloques internos
    \colorlet{innerblocktitlebgcolor}{colorThree}
    \colorlet{innerblocktitlefgcolor}{white}
    \colorlet{innerblockbodybgcolor}{white}
    \colorlet{innerblockbodyfgcolor}{black}
    % Colores de notas
    \colorlet{notefgcolor}{black}
    \colorlet{notebgcolor}{colorFour!15!white}
    \colorlet{notefrcolor}{colorFour!40!white}
 }

\definecolorstyle{Data}{
    \definecolor{colorOne}{HTML}{273746}
    \definecolor{colorTwo}{HTML}{A2D9CE}
    \definecolor{colorThree}{HTML}{C8512D}
    \definecolor{colorFour}{HTML}{AAB7B8}
}{
    % Colores de fondo
    \colorlet{backgroundcolor}{white}
    \colorlet{framecolor}{white}
    % Colores bloques
    \colorlet{blocktitlebgcolor}{colorTwo!35!white}
    \colorlet{blocktitlefgcolor}{colorOne}
    \colorlet{blockbodybgcolor}{white}
    \colorlet{blockbodyfgcolor}{black}
    % Colores de bloques internos
    \colorlet{innerblocktitlebgcolor}{colorThree}
    \colorlet{innerblocktitlefgcolor}{white}
    \colorlet{innerblockbodybgcolor}{white}
    \colorlet{innerblockbodyfgcolor}{black}
    % Colores de notas
    \colorlet{notefgcolor}{black}
    \colorlet{notebgcolor}{colorFour!15!white}
    \colorlet{notefrcolor}{colorFour!40!white}
 }

\definecolorstyle{Utah}{
    \definecolor{colorOne}{HTML}{283747}
    \definecolor{colorOne}{HTML}{AE0D45}
    \definecolor{colorThree}{HTML}{D1F2EB}
    \definecolor{colorFour}{HTML}{AAB7B8}
}{
    % Colores de fondo
    \colorlet{backgroundcolor}{white}
    \colorlet{framecolor}{white}
    % Colores bloques
    \colorlet{blocktitlebgcolor}{colorTwo!35!white}
    \colorlet{blocktitlefgcolor}{colorOne}
    \colorlet{blockbodybgcolor}{white}
    \colorlet{blockbodyfgcolor}{black}
    % Colores de bloques internos
    \colorlet{innerblocktitlebgcolor}{colorThree}
    \colorlet{innerblocktitlefgcolor}{black}
    \colorlet{innerblockbodybgcolor}{white}
    \colorlet{innerblockbodyfgcolor}{black}
    % Colores de notas
    \colorlet{notefgcolor}{black}
    \colorlet{notebgcolor}{colorFour!15!white}
    \colorlet{notefrcolor}{colorFour!40!white}
 }
 
\definecolorstyle{Elena} {
	\definecolor{colorOne}{RGB}{47,104,138}
}{
	% Colores de fondo
	\colorlet{backgroundcolor}{colorOne!20!white} % Fondo general
	\colorlet{framecolor}{black}						 % Color de marco (no utilizado)
	% Colores bloques
	\colorlet{blocktitlebgcolor}{colorOne}	% Fondo cabecera
	\colorlet{blocktitlefgcolor}{white}	% Texto cabecera
	\colorlet{blockbodybgcolor}{white}			% Fondo cuerpo
	\colorlet{blockbodyfgcolor}{black}			% Texto cuerpo
	% Colores de bloques internos
	\colorlet{innerblocktitlebgcolor}{colorOne}				% Borde
	\colorlet{innerblocktitlefgcolor}{white}				% Texto cabecera
	\colorlet{innerblockbodybgcolor}{white}	% Fondo
	\colorlet{innerblockbodyfgcolor}{black}				% Texto cuerpo
	% Colores de notas
	\colorlet{notefgcolor}{black}				% Texto
	\colorlet{notebgcolor}{yellow!50!white}		% Fondo
	\colorlet{noteframecolor}{yellow}			% Borde
}

%%%%%%%%%%%%%%%%%%%%%%%% 
% 2. ESTILO DE BLOQUES. 
%%%%%%%%%%%%%%%%%%%%%%%%
% Estilos disponibles:
% 'TFGTFM', 'Default', 'Basic', 'Minimal','Envelope', 'Corner', 
% 'Slide', 'TornOut', 'Barra',

% Comando para cargar el estilo elegido y guardarlo para otros usos
\newcommand{\estilobloque}[1]{
\useblockstyle{#1}
\def\estilo{#1}
}

\defineblockstyle{TFGTFM}{ % Modificación del estilo 'Slide'
    titlewidthscale=1, bodywidthscale=1, titleleft,
    titleoffsetx=0pt, titleoffsety=0pt, bodyoffsetx=0pt, bodyoffsety=0pt,
    bodyverticalshift=0pt, roundedcorners=0, linewidth=0pt, titleinnersep=1cm,
    bodyinnersep=1cm 
}{
	% Estilo cabecera
    \ifBlockHasTitle 
        \draw[draw=none, left color=blocktitlebgcolor, right color=blocktitlebgcolor!50!blockbodybgcolor]
           (blocktitle.north west) [rounded corners=0] -- (blocktitle.south west) --
        (blocktitle.south east) [rounded corners=5]-- (blocktitle.north east) -- cycle;
    	% Estilo contenido (cuando hay título)
    	\draw[draw=none, fill=blockbodybgcolor] %
        (blockbody.north west) [rounded corners=30] -- (blockbody.south west) --
        (blockbody.south east) [rounded corners=0]-- (blockbody.north east) -- cycle;
    \fi
    % Estilo contenido (haya o no título)
    \draw[draw=none, fill=blockbodybgcolor] %
        (blockbody.north west) [rounded corners=30] -- (blockbody.south west) --
        (blockbody.south east) [rounded corners=30]-- (blockbody.north east) -- cycle;
}

\defineblockstyle{Barra}{
    titlewidthscale=1, bodywidthscale=1, titleleft,
    titleoffsetx=0pt, titleoffsety=0pt, bodyoffsetx=5pt, bodyoffsety=0pt,
    bodyverticalshift=0pt, roundedcorners=0, linewidth=0.2cm,
    titleinnersep=1cm, bodyinnersep=1cm
}{
    \begin{scope}[line width=\blocklinewidth, rounded corners=\blockroundedcorners]
    	% Estilo cabecera (cuando hay título)
       	\ifBlockHasTitle
      		\draw[color=blocktitlefgcolor, line width = 10pt]
               ([xshift=30pt, yshift=5pt]blocktitle.south west) -- ([xshift=-30pt, yshift=5pt]blocktitle.south east);%
    	\else
    		% Estilo cabecera (cuando no hay título)
         	\draw[draw=none]%, fill=blockbodybgcolor]
                 (blockbody.south west) rectangle (blockbody.north east);
        \fi
    \end{scope}
}

%%%%%%%%%%%%%%%%%%%%%%%% 
% 3. ESTILO DE BLOQUES INTERNOS. 
%%%%%%%%%%%%%%%%%%%%%%%%
% Estilos disponibles:
% 'TFGTFM', 'Default', 'Table', 'Basic', 'Minimal', 
% 'Envelope', 'Corner', 'Slide', 'TornOut'

\defineinnerblockstyle{TFGTFM}{
    titlewidthscale=0.25, bodywidthscale=0.75, titlecenter,
    titleoffsetx=0pt, titleoffsety=0pt, bodyoffsetx=0pt, bodyoffsety=0pt,
    bodyverticalshift=0pt, roundedcorners=15, linewidth=3mm,
    titleinnersep=15pt, bodyinnersep=15pt
}{
  % minimum height should be the maximum of \TP@innerblocktitleheight and
  % \TP@innerblockbodyheight
  \node[minimum width=\TP@innerblocktitlewidth, minimum
  height=\TP@innerblockbodyheight, anchor=center] (innerblocktitle) at
  (\TP@innerblockcenter-0.5\TP@innerblockbodywidth+\TP@innerblocktitleoffsetx,
  {-\TP@innerblocktitleheight-0.5\TP@innerblockbodyheight+\TP@innerblocktitleoffsety})
  {};%
  %
  \ifInnerblockHasTitle%
  \node[minimum width=\TP@innerblockbodywidth, minimum
  height=\TP@innerblockbodyheight, anchor=center] (innerblockbody) at
  (\TP@innerblockcenter+0.5\TP@innerblocktitlewidth+\TP@innerblockbodyoffsetx,
  {-\TP@innerblocktitleheight-0.5\TP@innerblockbodyheight+\TP@innerblockbodyoffsety})
  {};%
  %
  \else%
  \node[minimum width=\TP@innerblockbodywidth, minimum
  height=\TP@innerblockbodyheight, anchor=center] (innerblockbody) at
  (\TP@innerblockcenter+\TP@innerblockbodyoffsetx,
  {-\TP@innerblocktitleheight-0.5\TP@innerblockbodyheight}) {};%
  \fi
 \begin{scope}[rounded corners=\innerblockroundedcorners, line width=\innerblocklinewidth]
        \ifInnerblockHasTitle
           % the big rectangle
        \draw[color=innerblocktitlebgcolor, fill=innerblockbodybgcolor]
        (innerblocktitle.north west) rectangle (innerblockbody.south east);%
        \draw[color=innerblocktitlebgcolor] (innerblocktitle.south east) --
        (innerblocktitle.north east); %
        \else
           % No title
           \draw[color=innerblocktitlebgcolor, fill=innerblockbodybgcolor]
               (innerblockbody.south west) rectangle (innerblockbody.north east);
        \fi
    \end{scope}
}


%%%%%%%%%%%%%%%%%%%%%%%% 
% 4. ESTILO DE FONDOS 
%%%%%%%%%%%%%%%%%%%%%%%%
% Estilos disponibles:
% 'TFGTFM', 'Rayos', 'Gradiente', 'GradienteInferior', 'Vacio'

\definebackgroundstyle{TFGTFM}{
    \fill[inner sep=0pt, line width=0pt, color=backgroundcolor]%
    (bottomleft) rectangle (topright);
}

\definebackgroundstyle{Rayos}{
    \draw[line width=0pt, top color=backgroundcolor!70, bottom
    color=backgroundcolor!70!black] (bottomleft) rectangle (topright);
    %
    \begin{scope}
        \foreach \a in {10,20,...,80}{%
            \draw[backgroundcolor, line width=0.15cm](bottomleft) --
            ($(bottomleft)!1!(bottomleft)+(\a:120)$);%
        }
        \foreach \i in {1,2,...,50}{%
            \begin{scope}[shift={($(rand*60,rand*70)$)}]
                \draw[backgroundcolor!50!, line width=0.1cm] (0,0) circle (4);
            \end{scope}
        }
    \end{scope}
}

\definebackgroundstyle{Gradiente}{
    \draw[line width=0pt, bottom color=backgroundcolor, top
     color=backgroundcolor!60!white] (bottomleft) rectangle (topright);
}

\definebackgroundstyle{GradienteInferior}{
    \draw[line width=0pt, bottom color=backgroundcolor!60!white, top
     color=backgroundcolor] (bottomleft) rectangle (topright);
}

\definebackgroundstyle{Vacio}{
  %
}

\definebackgroundstyle{Quadro}{
\shade[upper left=quadro1,upper right=quadro2,lower left=quadro3,lower right=quadro4] (bottomleft) rectangle (topright);
}
%%%
% Comando para incluir imagen de fondo
%%%
\newcommand{\imagenfondo}[2]{
\node[above right,opacity=#1,inner sep=0pt,outer sep=0pt] at (bottomleft) {\includegraphics[width=\paperwidth,height=\paperheight]{#2}};
}


%%%%%%%%%%%%%%%%%%%%%%%% 
% 5. ESTILO DE NOTAS 
%%%%%%%%%%%%%%%%%%%%%%%%
\definenotestyle{Default}{
    targetoffsetx=0pt, targetoffsety=0pt, angle=0, radius=8cm, width=8cm,
    connection=false, rotate=0, roundedcorners=20, linewidth=0pt, innersep=1cm
}{
    \ifNoteHasConnection %% callout note
        \draw[color=notefrcolor, fill=notebgcolor]%
         (notetarget) -- ($(notetarget)!1!4:(notecenter.center)$) --
         ($(notetarget)!1!-4:(notecenter.center)$) --cycle; %
         %
    \fi
    % the body of the note
    \draw[color=notefrcolor, fill=notebgcolor, rounded
    corners=\noteroundedcorners] (notecenter.south west) -- (notecenter.north
    west) -- (notecenter.north east) -- (notecenter.south east) -- cycle;
}

 \definenotestyle{Corner}{
    targetoffsetx=0pt, targetoffsety=0pt, angle=0, radius=8cm, width=12cm,
    connection=false, rotate=0, roundedcorners=20, linewidth=0pt, innersep=1cm
}{
    \ifNoteHasConnection % callout note
      \draw[color=notebgcolor, fill=notebgcolor, drop shadow={shadow
        xshift=0.2cm, shadow yshift=-0.2cm, opacity=0.3}] %
        (notetarget) -- ($(notetarget)!1!4:(notecenter.center)$) --
         ($(notetarget)!1!-4:(notecenter.center)$) --cycle; %
    \fi
    % the body of the note
    % the shape
    \def \border{%
        [rounded corners=0] (notecenter.south west) -- (notecenter.north west) %
        [rounded corners=\noteroundedcorners] -- ($(notecenter.north
        east)-(\noterotate:4.7)$) %
        [rounded corners=\noteroundedcorners] -- ($(notecenter.north
        east)+(-90+\noterotate:1.7)$) %
        [rounded corners=0] -- (notecenter.south east) -- (notecenter.south
        west) -- cycle%
   }
    \fill[color=notebgcolor] \border;
    \coordinate (x) at (\noterotate:1);
    \coordinate (y) at (\noterotate-90:1);
    % the shadow of the corner
    \fill[color=gray,opacity=0.3] ($(notecenter.north east)+3*(y)$) --
        ($(notecenter.north east)+2.5*(y)$) .. %
        controls ($(notecenter.north east)+1.25*(y)$) and ($(notecenter.north
        east)-1.5*(x)+1.25*(y)$) .. %
        ($(notecenter.north east)-1.9*(x)+2.5*(y)$) .. %
        controls ($(notecenter.north east)-4.5*(x)$) .. %
        ($(notecenter.north east)-5.7*(x)$) %
        [rounded corners=\noteroundedcorners] -- ($(notecenter.north east)-4.7*(x)$) %
        [rounded corners=\noteroundedcorners] -- ($(notecenter.north east)+1.7*(y)$) %
        [rounded corners=0] -- ($(notecenter.north east)+3*(y)$);
    % the corner
    \fill[color=notefrcolor] %
        ($(notecenter.north east)+3*(y)$) -- ($(notecenter.north east)+2.5*(y)$) .. %
        controls ($(notecenter.north east)+1.25*(y)$) and ($(notecenter.north
        east)-1.5*(x)+1.25*(y)$) .. %
        ($(notecenter.north east)-1.9*(x)+2.3*(y)$) .. %
        controls ($(notecenter.north east)-4.5*(x)$) .. %
        ($(notecenter.north east)-5.7*(x)$) %
        [rounded corners=\noteroundedcorners] -- ($(notecenter.north east)-4.7*(x)$) %
        [rounded corners=\noteroundedcorners] -- ($(notecenter.north east)+1.7*(y)$) %
        [rounded corners=0] -- ($(notecenter.north east)+3*(y)$);
}

 \definenotestyle{Gradiente}{
    targetoffsetx=0pt, targetoffsety=0pt, angle=0, radius=8cm, width=8cm,
    connection=false, rotate=0, roundedcorners=20, linewidth=1pt, innersep=1cm
}{
    \ifNoteHasConnection % callout note
         % the shadow
         \begin{scope}[opacity=0.3]
            \begin{pgftransparencygroup}
              \coordinate (shadowshift) at (0.2cm,-0.2cm); \fill%
              ($(notetarget)+(shadowshift)$) --
              ($(notetarget)!1!4:(notecenter.center)+(shadowshift)$) --
              ($(notetarget)!1!-4:(notecenter.center)+(shadowshift)$) --cycle; %
              \fill[rounded corners=\noteroundedcorners] %
              ($(notecenter.south west)+(shadowshift)$) -- ($(notecenter.north
              west)+(shadowshift)$) -- ($(notecenter.north east)+(shadowshift)$)
              -- ($(notecenter.south east)+(shadowshift)$) -- cycle;
            \end{pgftransparencygroup}
          \end{scope}
          %% the main drawing
          %
          %% the border
          \draw[color=notefrcolor, line width=\notelinewidth*2]%
          (notetarget) -- ($(notetarget)!1!4:(notecenter.center)$) --
          ($(notetarget)!1!-4:(notecenter.center)$) -- cycle;%
          \draw[color=notefrcolor, line width=\notelinewidth*2, rounded
          corners=\noteroundedcorners]%
          (notecenter.south west) -- (notecenter.north west) --
          (notecenter.north east) -- (notecenter.south east) -- cycle; %
          %
          %% the filling (vertical shading), shared between the note and the connection
          \begin{scope}
            \node[fit=(notetarget)(notecenter.south west)(notecenter.south east)
            (notecenter.north east) (notecenter.north west), inner sep=+0pt]
            (box) {};%
            %
            \clip (notetarget) -- ($(notetarget)!1!4:(notecenter.center)$) --
            ($(notetarget)!1!-4:(notecenter.center)$) -- cycle%
            [rounded corners=\noteroundedcorners] (notecenter.south west) --
            (notecenter.north west) -- (notecenter.north east) --
            (notecenter.south east) -- cycle;
            %
            \draw[draw=none, color=notefrcolor, top color=notebgcolor!60, bottom
            color=notebgcolor] %
            (box.south west) rectangle (box.north east);
          \end{scope}
          %
    \else % the simple note
        \begin{scope}[drop shadow={shadow xshift=0.2cm, shadow yshift=-0.2cm,
           opacity=0.3}]
         \draw[line width=\notelinewidth, rounded corners=\noteroundedcorners,
         color=notefrcolor, top color=notebgcolor!60, bottom color=notebgcolor,
         drop shadow] %
         (notecenter.south west) -- (notecenter.north west) -- (notecenter.north
         east) -- (notecenter.south east) -- cycle;
        \end{scope}
    \fi
}

 \definenotestyle{Sticky}{
    targetoffsetx=0pt, targetoffsety=0pt, angle=0, radius=8cm, width=8cm,
    connection=false, rotate=0, roundedcorners=0, linewidth=0pt, innersep=1cm
}{
    \ifNoteHasConnection %% callout note
    \draw[color=notefrcolor, fill=notebgcolor, drop shadow={shadow
        xshift=0.2cm, shadow yshift=-0.2cm, opacity=0.3}] %
         (notetarget) -- ($(notetarget)!1!4:(notecenter.center)$) --
         ($(notetarget)!1!-4:(notecenter.center)$) --cycle; %
    \fi
    % the body of the note
    % shadow
    \draw[draw=none, fill=gray, opacity=0.3]
        ($(notecenter.north east)+(-0.5,0)$) [rounded corners=40]--%
        (notecenter.north west) [rounded corners=0] -- %
        ($(notecenter.south west)$) .. %
        controls ($0.2*(notecenter.south west) + 0.8*(notecenter.south east)$) .. %
        ($(notecenter.south east)+(-0.2,0.3)$) .. %
        controls ($0.75*(notecenter.south east) + 0.25*(notecenter.north east) - (0.5,0)$) .. %
        ($(notecenter.north east)+(-0.5,0)$);
    % the shape
    \def \border{%
        ($(notecenter.north east)+(-0.5,0)$) [rounded corners=40]--%
        (notecenter.north west) [rounded corners=0] -- %
        ($(notecenter.south west)$) .. %
        controls ($0.2*(notecenter.south west) + 0.8*(notecenter.south east)$) .. %
        ($(notecenter.south east)+(0,0.7)$) .. %
        controls ($0.75*(notecenter.south east) +0.25*(notecenter.north east) -(0.5,0)$) .. %
        ($(notecenter.north east)+(-0.5,0)$)%
    }%
    \draw[color=notefrcolor, fill=notebgcolor]
    \border;
    % the shading in the left top corner
    \begin{scope}
        \clip \border; %
        \begin{scope}[transform canvas={rotate
            around={\noterotate+15:(notecenter.north west)}}]
            \fill[notebgcolor!60!black, path fading=south, opacity=0.6]%
                (notecenter.north west) -- +(-3,0) |- ($(notecenter.north west) + (0,-1.2)$)
                -- ($(notecenter.north west) + (4,-1.2)$) |- ($(notecenter.north west)$);
        \end{scope}
    \end{scope}
}

%%%%%%%%%%%%%%%%%%%%%%%% 
% 6. ELEMENTOS GRÁFICOS
%%%%%%%%%%%%%%%%%%%%%%%%

%%%%%%%%%%%%
% 6.1. Modificadores de bloques
%%%%%%%%%%%%

% Flecha entre cajas
\newcommand{\flechacaja}[3]{
% Rotación de la flecha y altura
\ifthenelse{\equal{#2}{south}}{\def\rotacion{-90},\def\altura{\TP@blockverticalspace}}{
\ifthenelse{\equal{#2}{west}}{\def\rotacion{180},\def\altura{\TP@colspace}}{
\ifthenelse{\equal{#2}{north}}{\def\rotacion{90},\def\altura{\TP@blockverticalspace}}{
\ifthenelse{\equal{#2}{east}}{\def\rotacion{0},\def\altura{\TP@colspace}}
}}}

% Color de la flecha
\ifthenelse{\isempty{#3}}{\def\colorflecha{blockbodybgcolor}}{\def\colorflecha{#3}}

% Estilos de flecha

\ifthenelse{\equal{\estilo}{TFGTFM}}{
\begin{scope}[rotate=\rotacion]
\fill[color=\colorflecha] 
		([shift={(-2mm,-2cm)}]#1.#2) -- 
		([shift={(0.7\altura,0)}]#1.#2) -- 
		([shift={(-2mm,2cm)}]#1.#2)--
		([shift={(-2mm,-2cm)}]#1.#2);
\end{scope}
}

\ifthenelse{\equal{\estilo}{Default}}{
\ifthenelse{\isempty{#3}}{\def\colorflecha{blocktitlebgcolor}}{\def\colorflecha{#3}}
\begin{scope}[rotate=\rotacion]
\fill[color=\colorflecha] 
		([shift={(-2mm,-2cm)}]#1.#2) -- 
		([shift={(0.7\altura,0)}]#1.#2) -- 
		([shift={(-2mm,2cm)}]#1.#2)--
		([shift={(-2mm,-2cm)}]#1.#2);
\end{scope}
}

\ifthenelse{\equal{\estilo}{Basic}}{
\begin{scope}[rotate=\rotacion]
\fill[color=\colorflecha] 
		([shift={(-1.8\blocklinewidth,-2cm)}]#1.#2) -- 
		([shift={(0.7\altura,0)}]#1.#2) -- 
		([shift={(-1.8\blocklinewidth,2cm)}]#1.#2)--
		([shift={(-1.8\blocklinewidth,-2cm)}]#1.#2);
\draw[color=framecolor,line width=\blocklinewidth] 
		([shift={(-1.2\blocklinewidth,-2cm)}]#1.#2) -- 
		([shift={(0.7\altura,0)}]#1.#2) -- 
		([shift={(-1.2\blocklinewidth,2cm)}]#1.#2);
\end{scope}
}

\ifthenelse{\equal{\estilo}{Envelope}}{
\begin{scope}[rotate=\rotacion]
\fill[color=\colorflecha] 
		([shift={(-4\blocklinewidth,-2cm)}]#1.#2) -- 
		([shift={(0.7\altura,0)}]#1.#2) -- 
		([shift={(-4\blocklinewidth,2cm)}]#1.#2)--
		([shift={(-4\blocklinewidth,-2cm)}]#1.#2);
\draw[color=blocktitlebgcolor,line width=\blocklinewidth] 
		([shift={(-3.5\blocklinewidth,-2cm)}]#1.#2) -- 
		([shift={(0.7\altura,0)}]#1.#2) -- 
		([shift={(-3.5\blocklinewidth,2cm)}]#1.#2);
\end{scope}
}

\ifthenelse{\equal{\estilo}{Corner}}{
\begin{scope}[rotate=\rotacion]
\fill[color=\colorflecha] 
		([shift={(-5\blocklinewidth,-2cm)}]#1.#2) -- 
		([shift={(0.7\altura,0)}]#1.#2) -- 
		([shift={(-5\blocklinewidth,2cm)}]#1.#2)--
		([shift={(-5\blocklinewidth,-2cm)}]#1.#2);
\draw[color=blocktitlebgcolor,line width=\blocklinewidth] 
		([shift={(-4.5\blocklinewidth,-2cm)}]#1.#2) -- 
		([shift={(0.7\altura,0)}]#1.#2) -- 
		([shift={(-4.5\blocklinewidth,2cm)}]#1.#2);
\end{scope}
}

\ifthenelse{\equal{\estilo}{Slide}}{
\begin{scope}[rotate=\rotacion]
\fill[color=\colorflecha] 
		([shift={(-2mm,-2cm)}]#1.#2) -- 
		([shift={(0.7\altura,0)}]#1.#2) -- 
		([shift={(-2mm,2cm)}]#1.#2)--
		([shift={(-2mm,-2cm)}]#1.#2);
\end{scope}
}

\ifthenelse{\equal{\estilo}{TornOut}}{
\begin{scope}[rotate=\rotacion]
\fill[color=\colorflecha] 
		([shift={(-8\blocklinewidth,-2cm)}]#1.#2) -- 
		([shift={(0.7\altura,0)}]#1.#2) -- 
		([shift={(-8\blocklinewidth,2cm)}]#1.#2)--
		([shift={(-8\blocklinewidth,-2cm)}]#1.#2);

\end{scope}
}
}

%%%%%%%%%%%%
% 6.2. Flechas
%%%%%%%%%%%%

% Con flecha en ambos extremos
\newcommand{\flechaA}[7]{
\draw [stealth-stealth, line width=12mm,postaction={decorate,decoration={raise=#1,text along path,text align=center,text={|\color{#2}\bfseries|#3}}},#4] (#5) to[#6] (#7);}
% Con flecha solo en el extremo final
\newcommand{\flechaB}[7]{
\draw [-stealth, line width=12mm,postaction={decorate,decoration={raise=#1,text along path,text align=center,text={|\color{#2}\bfseries|#3}}},#4] (#5) to[#6] (#7);}



% Para mostrar la fecha actual (mes año) con \Hoy
\newcommand{\MES}{%
  \ifcase\month% 0
    \or Enero% 1
    \or Febrero% 2
    \or Marzo% 3
    \or Abril% 4
    \or Mayo% 5
    \or Junio% 6
    \or Julio% 7
    \or Agosto% 8
    \or Septiembre% 9
    \or Octubre% 10
    \or Noviembre% 11
    \or Diciembre% 12
  \fi}
\newcommand{\ANYO}{\number\year}
\newcommand{\Hoy}{\MES\ \ANYO}





