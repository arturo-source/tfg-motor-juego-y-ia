%%%%%%%%%%%%%%%%%%%%%%%%%%%%%%%%%%%%%%%%%%%%%%%%%%%%%%%%%%%%%%%%%%%%%%%%
% Plantilla TFG/TFM
% Escuela Politécnica Superior de la Universidad de Alicante
% Realizado por: Jose Manuel Requena Plens
% Contacto: info@jmrplens.com / Telegram:@jmrplens
%%%%%%%%%%%%%%%%%%%%%%%%%%%%%%%%%%%%%%%%%%%%%%%%%%%%%%%%%%%%%%%%%%%%%%%%

\chapter{Conclusiones}
\label{conclusiones}

Para finalizar el \gls{tfg} voy a nombrar las conclusiones a las que he llegado. Tras estos meses de desarrollo del proyecto, voy a poner en tela de juicio las decisiones tomadas desde el comienzo. 

La primera decisión fue desarrollar yo mismo el motor del videojuego, en lugar de usar uno ya existente, evitando el uso de librerías externas para conocer a fondo cómo se crea un motor de videojuegos. Como experiencia de aprendizaje, ha sido algo fructífero, pero al mismo tiempo me ha quitado tiempo de desarrollo. No solo hay que tener en cuenta que hasta noviembre no terminé el motor, sino que el hecho de hacerlo yo mismo ha llevado a tener que modificar partes del código para que funcione en otros sistemas operativos, tener que cambiar el sistema de ``Render'' puesto que para hacer los menús me convenía poder usar otras herramientas, etc. Todos estos inconvenientes que sumados forman una parte importante del desarrollo de este proyecto, no habrían surgido y ese tiempo del desarrollo podría haber sido destinado a ampliar otras partes del proyecto que se han quedado con carencias, que posteriormente mencionaré.
\\
Por lo tanto, las ventajas son claras, el objetivo era descubrir cómo se hacía un motor de videojuegos desde cero, y eso he hecho, comprendiendo y explicándolo en el apartado de desarrollo (apartado \ref{desarrollo}). Los inconvenientes han sido mayormente uno, el tiempo. Por supuesto, otro inconveniente a destacar podría ser que los motores ya existentes te ofrecen herramientas para que el juego quede más profesional, pero como ya he mencionado anteriormente, el objetivo del proyecto no era tener un juego con acabados profesionales.

Respecto a la optimización de la caché, utilizando los resultados que obtuve al hacerlo de forma aislada con test creados por mí, me di cuenta de que los ordenadores están tan avanzados que son demasiado complejos como para determinar con mis conocimientos qué es más eficiente y qué no. Como has visto en el apartado de resultados (apartado \ref{resultados}), finalmente demostré que leer la memoria linealmente era más rápido, pero el primer test que pensaba que afirmaría este pretexto fue fallido. Por lo que soy incapaz de decir si verdaderamente las decisiones de optimización de caché tomadas durante el desarrollo del proyecto han sido beneficiosas o no, para ello habría que desarrollar un sistema distinto de organización de entidades y componentes en memoria, y de esta manera compararlo con el ya existente, para demostrar empíricamente que fueron decisiones bien tomadas.

El juego ha terminado conforme se planteó al principio, un juego en el que puedes entrenar a un agente y ponerlo a jugar contra otros agentes. Sin embargo, no he terminado de añadir todas las funciones que quería en él. Debido a que el tiempo del desarrollo del \gls{tfg} está a punto de terminar, he tenido que limitar las características del juego para poder entregar el \gls{tfg} dentro de plazo. Aún así, ha terminado siendo algo de lo que estar orgulloso.

En cuanto a la \gls{ia}, he llegado a la conclusión de que el perceptrón resolvía mejor el problema de jugar a este juego que la red neuronal. Puede ser, en parte, porque es un juego bastante simple que sólo consta de pulsar abajo o arriba, por lo tanto el perceptrón entrenado de forma aleatoria, con la poca cantidad de pesos que tiene, puede encontrar una solución funcional. La red neuronal, al ser un sistema más complejo, con un mayor número de términos independientes aprende mediante backpropagation, pero puede que haya demasiado ruido en las muestras, y por eso no sea el algoritmo más apropiado para hacer aprender a la red. Sin embargo, es un punto más interesante para el usuario final, poder ajustar los parámetros del entrenamiento de la \gls{ia}, por eso he decidido dejar la red neuronal dentro del juego, porque es más interesante ajustarlos tú mismo, que simplemente apretar un botón y ver cómo avanza el entrenamiento aleatorio sin ajustar nada.