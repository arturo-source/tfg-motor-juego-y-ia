%%%%%%%%%%%%%%%%%%%%%%%%%%%%%%%%%%%%%%%%%%%%%%%%%%%%%%%%%%%%%%%%%%%%%%%%
% Plantilla TFG/TFM
% Escuela Politécnica Superior de la Universidad de Alicante
% Realizado por: Jose Manuel Requena Plens
% Contacto: info@jmrplens.com / Telegram:@jmrplens
%%%%%%%%%%%%%%%%%%%%%%%%%%%%%%%%%%%%%%%%%%%%%%%%%%%%%%%%%%%%%%%%%%%%%%%%

\chapter{Objetivos}
\label{objetivos} 
He de remarcar algo más abstracto sobre los objetivos del proyecto en sí, no del juego: la mayor parte del software usado será de desarrollo propio, es decir, intentaré minimizar el uso de librerías de terceros (sólo usándolas para tareas menos relevantes en el proyecto). Así que el motor del videojuego, el videojuego y la \gls{ia} en sí serán desarrollados desde cero en el lenguaje C/C++. Esto es así porque el objetivo final no es obtener una productividad absoluta, utilizando tecnologías existentes que me permitirían terminar el proyecto en una cantidad de tiempo mucho menor, sino que tengo que estudiar estas tecnologías desde la raíz, analizarlas, y entender en profundidad cómo funcionan estos tres campos en los que se centra mi \gls{tfg}. De esta manera trabajaré en el conocimiento de cómo está hecha cada tecnología y en un futuro, aportar desde las bases de cada una de ellas.

Más concretamente, los objetivos son:
\begin{itemize}
	\item Estudiar tecnologías de motor de videojuegos e implementarlas en el proyecto.
	\item Conocer cuáles son las librerías y tecnologías mínimas para desarrollar el videojuego.
	\item Analizar la evolución del uso de la \gls{ia} en el mundo de los videojuegos.
	\item Analizar el estado del arte actual del \gls{ml} en videojuegos.
	\item Diseñar e implementar prototipos simples de mecánicas de juego.
	\item Diseñar e implementar técnicas de \gls{ml} para que les sea fácilmente utilizable a los jugadores.
	\item Diseñar métricas para poder medir si la \gls{ia} ha cumplido con los objetivos.
	\item Crear una interfaz para que estas métricas sean visibles ajustables por los jugadores reales.
	\item Crear un modo de juego para jugador contra jugador.
	\item Diseñar diferentes modos de juego en los que el jugador real pueda entrenar a su \gls{ia} y captar los datos necesarios para el entrenamiento.
	\item Crear un modo de juego en el que dos jugadores aporten su \gls{ia} y se visualice el enfrentamiento.
\end{itemize}