%%%%%%%%%%%%%%%%%%%%%%%%%%%%%%%%%%%%%%%%%%%%%%%%%%%%%%%%%%%%%%%%%%%%%%%%
% Plantilla TFG/TFM
% Escuela Politécnica Superior de la Universidad de Alicante
% Realizado por: Jose Manuel Requena Plens
% Contacto: info@jmrplens.com / Telegram:@jmrplens
%%%%%%%%%%%%%%%%%%%%%%%%%%%%%%%%%%%%%%%%%%%%%%%%%%%%%%%%%%%%%%%%%%%%%%%%

\chapter{Introducción}
\label{introduccion}

\section{Historia}
El mundo de los videojuegos va, en muchas ocasiones, de la mano con el \gls{ml}, debido a que es difícil desarrollar un algoritmo en los juegos en los que tienes que enfrentarte contra la máquina, y además, que este se ajuste a diferentes niveles de dificultad, para poder ajustarla según la habilidad del jugador. Sin entrar en profundidad de lo que hablaremos en los siguientes apartados, porque contaremos la historia de los motores, los videojuegos, y la \gls{ia}, he de mencionar algunos ejemplos de juegos emblemáticos que ilustren cómo ha sido la progresión a lo largo del tiempo. 
\begin{itemize}
	\item Tetris, es un juego del tipo puzle, que fue lanzado en 1984. Este juego no requiere de \gls{ia} ya que no tiene enemigos, solo has de tratar de no dejar huecos colocando piezas con distintas formas que aparecen en la parte más alta de la pantalla de con una de las formas aleatoria. 
	\item Mario Bros, es un juego de tipo plataformas. Fue lanzado en 1983, en su primera versión, pero gracias al éxito que tuvo le siguió una larga saga de continuaciones del mismo y otras versiones con el mismo personaje principal y secundarios en las que el juego no trataba de lo mismo. En este juego sí que te encontrarás con personajes, has de enfrentarte a ellos, pero como el atractivo principal del juego es llegar a la meta moviéndote, la inteligencia de los mismos no requiere de ser muy compleja. Todos los personajes que aparecen tienen un comportamiento programado, como moverse de lado a lado, salir de una tubería después de cierta cantidad de segundos, etc. Lo importante es que, aún con la presencia de enemigos, no era necesario una \gls{ia}.
	\item Pokemon Red, Pokemon Green y Pokemon Blue. Los tres fueron lanzados en 1996, y son prácticamente iguales pero con sutiles diferencias. Estos juego son del tipo \gls{rpg}, es decir, que te introduce en la vida de un personaje, y has de jugar como si fueras tú el que toma y sufre las decisiones tomadas. A diferencia de Mario Bros, que tenías que moverte en una sola dirección y saltar, aquí te moverás en dos direcciones, y por lo tanto, los \gls{npc} también, pero estos siguen sin presentar ningún tipo de inteligencia avanzada, ya que normalmente siguen patrones en su camino que ya han sido programados, y los ataques que lanzan los Pokemon enemigos en los combates parecen ser aleatorios. 
	\item Left 4 Dead es un juego de supervivencia, y fue lanzado en 2008. Se trata de un juego de zombies en el cual vas con un equipo de cuatro personas, es aquí donde entra la \gls{ia}. Puedes jugar con tres amigos más, pero el juego también está hecho para jugar solo, por lo que los otros personajes requieren una inteligencia para ayudarte a completar los desafíos. No se conoce a ciencia cierta cómo está programado, pero se puede deducir que los \gls{npc} no tienen inteligencia para la búsqueda de rutas(pathfinding es el término más conocido) porque las rutas ya están definidas en el juego, pero sí cuando disparar, salvarte, etc.  mientras que los enemigos parecen comportarse siempre de la misma manera. Por lo que Left 4 Dead es el primer juego que tiene una \gls{ia} compleja de entre los que he mencionado hasta ahora.
	\item League of Legends es un juego más actual, jugable en PC y fue lanzado en 2009. Pero además, ha sido lanzada una versión independiente para móviles en el año 2020. Este juego se trata de un \gls{moba}, en el cual te enfrentas con tus compañeros, contra otros cuatro oponentes y has de derrotar la torre principal del enemigo para ganar, teniendo como obstáculos no solo a los oponentes, sino también a minions y las propias torres que se encargan de dispararte. Podría parecer que no hay ninguna \gls{ia}, pero lo cierto es que sí que se aplica un campo que es el de la búsqueda de la ruta más rápida. Esto está presente en muchos otros juegos, y es cierto que es un campo de la \gls{ia}, pero no requiere de aprendizaje, por lo que no ha de ser entrenado con técnicas de \gls{ml} y no es necesario programar redes neuronales para el uso de la misma. Esto es usado tanto por los minions, para moverse por el mapa, como por el jugador, que se mueve haciendo clicks donde quiere ir, y no con AWSD como es típico en los videojuegos más actuales.
\end{itemize}

De todos los juegos mencionados anteriormente, no se puede saber si fueron creados utilizando un motor de videojuegos, salvo si los autores de los mismos desvelasen los detalles, pero es fácil de deducir que los más antiguos no utilizaron un motor, sino que fueron creados desde cero, pero los que tuvieron sagas reutilizarían parte del código, con lo que muchas de las partes del código las acabarían integrando en un motor, para la posterior creación de los siguientes juegos de la saga. Mientras que todos los juegos actuales utilizan un motor, debido a que ya existen motores muy buenos, y gastar el recursos del desarrollo en algo que ya existe haría perder beneficios a la empresa desarrolladora del videojuego.

\subsection{Motor de videojuegos}
Después de la creación de los primeros juegos, los programadores de videojuegos se dieron cuenta de que había partes del desarrollo que se repetían. Algunas como la creación y destrucción de entidades entre otras cosas. Por eso, las empresas dedicadas a ello, comenzaron a diferenciar la parte de la creación del juego con la del motor, ya que, con un mismo motor, se podrían hacer distintos videojuegos.
\\
Pero fue en los años 90, con la creación de videojuegos en 3D donde se acuñó el termino de ``motor gráfico''. Fue debido a que en la mayoría de los casos, el motor no solo se encargaba de la gestión de entidades sino también de la renderización del juego, así que de esta forma se entendía mejor para qué servía esta pieza de software. El que se conoce como primer motor gráfico para PC fue Freescape Engine, que se utilizaba para hacer juegos de disparos en primera persona, esto es debido a que los primeros juegos en popularizarse para PC fueron de este género.

Realmente los motores gráficos no solo sirven para crear juegos, así que con el tiempo, estos motores se empezaron a utilizar no solo para videojuegos, sino también para desarrollar aplicaciones que tengan que ver con simulación de la vida real, como pueden ser aplicaciones arquitectónicas, educación, simuladores para aprender a conducir, y un largo etcétera.

\subsection{Videojuegos}
Fueron creados como entretenimiento, mediante el uso de un hardware que actúa como entrada en el juego los interpreta y el juego actúa según la lógica programada. Los primeros datan de los año 1952 y 1958 (Tres en raya y Tennis for Two respectivamente). Pero con el paso de los años han progresado, llegando a ser más complejos. 
\\
Con el auge de las consolas, muchas empresas surgieron y crearon videojuegos, como Nintendo con su mundialmente conocido Mario Bros, Game Freak con Pokemon, y un largo etcétera. 
\\
Después de varios años, y la creación de motores de videojuegos, la gente empezó a poder tener ordenadores en casa. Surgen distintas tecnologías como Flash Player para navegadores, con lo que alguien con conocimientos de programación podía crear su videojuego sin preocuparse del renderizado, y compartirlo con el resto de usuarios de ordenador que tuvieran esta tecnología instalada en su navegador.

Conforme fue haciéndose más grande la industria de los videojuegos, algunas empresas empezaron a comercializar su motor de videojuegos, no solo vendiéndolo a otras grandes empresas que pudieran comprarlos, sino a pequeños desarrolladores, llamados desarrolladores indie\footnote{El término proviene de desarrollador independiente.}, los cuales no pueden permitirse costes tan altos. Pero si una empresa que desarrolla el motor consigue vender licencias más baratas a una cantidad muy grande de gente, hace que también sea rentable el desarrollo de motores de videojuegos. A causa de esto, y cuando Steam permitió vender juegos a los desarrolladores indie en su plataforma, aumentó la cantidad de videojuegos en el mercado.

Otra industria que ha crecido en los últimos años de manera exponencial son los videojuegos para móvil. A pesar de que con los primeros móviles ya se podía jugar a juegos más sencillos que ya venían preinstalados, como el Snake, el verdadero auge ha sido con la popularización de los sistemas operativos Android e iOS. Estos dos sistemas operativos son los que dominan en la actualidad todo el mercado de los móviles, y vienen con un marketplace instalado, cada uno el suyo. Tanto la Play Store (de Android) como la App Store (de iOS) permiten a cualquier desarrollador subir sus propios videojuegos\footnote{Para subir cualquier aplicación a una de ellas has de cumplir unos requisitos, ya que es un servicio centralizado que tiene que pasar unos estándares de seguridad para que no se cuelen virus o programas no deseados.}, con lo que se expandió aún más el mercado de los videojuegos. De hecho, el mundo de los videojuegos móviles es tan famoso que algunos motores de videojuegos te permiten compilar tu juego tanto para móviles como para ordenador, pudiendo de esta manera, rentabilizar aún más tu juego al abrirte a un mercado mayor, y esto ha hecho que aún más gente se anime a dedicarse a ser desarrollador.

Aunque actualmente, en el año 2021, la mayoría de juegos son construidos tanto para consola como para ordenador, por lo que con un ordenador puedes jugar casi todos, al principio se diseñaban máquinas especialmente para jugar, véase Atari 2600, NES, Game Boy, Mega Drive, Nintendo 64. Es decir, los juegos se hacían especializados para el hardware de la máquina, sin embargo, la popularización de los ordenadores domésticos hizo que cada vez haya que hacer los juegos más genéricos, y si una máquina en concreto necesita de una tecnología distinta, el motor se encarga de usarla a la hora de la compilación.


\subsection{\glsentrylong{ia}}
Cuando hablamos de \gls{ia} nos referimos a máquinas imitando comportamientos inteligentes. Esta definición es dada en 1955 con el nacimiento de la \gls{ia}. Pero si hablamos de \gls{ia} podemos estar hablando de distintos campos, ya que es un término muy genérico. Más concretamente, en este \gls{tfg} tratamos el campo del \gls{ml}.
\\
Para comprender la diferencia entre un comportamiento programado y un comportamiento aprendido me gustaría citar a Stuart J. Russell y Peter Norvig, \textit{"un agente aprende cuando su desempeño mejora con la experiencia; es decir, cuando la habilidad no estaba presente en su genotipo o rasgos de nacimiento"}.\footnote{Cita escrita en el libro \textit{Inteligencia Artificial: Un Enfoque Moderno} de los autores anteriormente mencionados.} Por lo tanto, podremos diferenciar dentro del mundo de la \gls{ia} (que es una lógica aplicada por una máquina), al \gls{ml} (que es un comportamiento, que además de ser aplicado por una máquina, ha de ser aprendido). Para más información sobre \gls{ml} vea la página \pageref{marcoteorico}.

La primera tecnología que permitía a una máquina tener \gls{ia} fue el perceptrón, presentado en 1959 por Rosenblatt. El perceptrón resuelve muy bien problemas generales, y de forma perfecta problemas que sean linealmente separables, como puede ser una puerta OR o una puerta AND, pero pronto se dieron cuenta de sus limitaciones, dado que una puerta lógica XOR no es linealmente separable, el perceptrón no funcionaba de forma correcta en estos casos. Este problema se solucionaría usando un segundo perceptrón, por lo tanto la lógica del programa depende de dos salidas, y sí que se podría separar.
\\
Pero el perceptrón no es el único modelo dentro del mundo de la \gls{ia}, en 1940 Warren McCulloch y Walter Pitts crearon un modelo informático que se llamaba lógica umbral, esto es lo primero que se conoce sobre redes neuronales. Si bien, su lógica es parecida a la del perceptrón, tiene algunas diferencias. 

Dada la lógica del perceptrón, y resuelta la problemática de los problemas no separables linealmente, se crearon las redes de perceptrones, también llamadas \gls{mlp}. Con ellas se podían resolver problemas como el de la puerta XOR, pero pronto se dieron cuenta de sus limitaciones, y es que al final, el hecho de aplicar multiples funciones aritméticas f(x) con sumas y multiplicaciones en cadena era equivalente a hacer una sola función aritmética f(x). Pero esto fue solucionado fácilmente cambiando la función de activación. Mientras que la función de activación de un perceptrón era una función escalonada, la función de activación de una neurona puede ser cualquier función matemática, las más comunes son la sigmoide, la ReLU, y la tangente hiperbólica. Esto es la principal diferencia entre un perceptrón y una neurona.
\\
Las limitaciones de las redes neuronales salieron a la luz después de hacer experimentos, ya que era demasiado costoso computacionalmente saber qué neurona dentro de la red neuronal había cometido el error, puesto que el método de búsqueda era por fuerza bruta, y esto supuso un gran parón en el mundo de la \gls{ia} y además porque el poder computacional en aquellos tiempos era más limitado.
\\
Pero en 1980 se inventó un algoritmo que era capaz de, en función de un error dado en la salida de la red neuronal, propagar el error hacia las neuronas de capas anteriores, de forma que cada una corregía su peso en la red neuronal en función de lo culpable que era en ese error. Su nombre es retropropagación, o como se le conoce por su nombre en inglés ``Backpropagation''. 

\section{Actualidad}
Actualmente, en el mundo hay muchos motores de videojuegos, algunos públicos de pago (Unreal Engine, Unity, etc.), utilizado sobretodo por desarrolladores indie, otros privados para que una empresa lo utilice o reutilice para hacer sus videojuegos, e incluso cabe mencionar Godot Engine por ser el más famoso entre los motores de código libre. En general, el mundo de los motores es un mundo bastante maduro, aunque de vez en cuando surjan mejoras.

Y con tanta oferta de motores, es obvio que estos existen porque el mundo de los videojuegos se ha ampliado, hasta tal punto que existen de todo tipo y para una gran cantidad de plataformas. En ordenadores existen plataformas de venta de videojuegos como Steam o Epic Games entre otros, donde puedes encontrar infinidad de títulos, entre los que me gustaría destacar Minecraft, Hollow Knight y Braid, demostrando que en el mundo de los videojuegos no hace falta ser enormes multinacionales para hacer juegos tremendamente entretenidos y con mecánicas originales. Mientras que en plataforma móvil, la mayoría de juegos los adquirirás en el Market Place del sistema operativo que use tu móvil, y por lo general, el mercado está copado de juegos casual con los que pasar un rato, pero también hay algunas joyitas, ya que las empresas de videojuegos se han dado cuenta de que es un mercado en crecimiento, Monument Valley, Pixel Dungeon o juegos ya famosos en PC que han portado su juego a versión móvil como League of Legends, Call of Duty o Mario Kart.

Mientras que el mundo de las \gls{ia}, aunque existen sistemas muy complejos y completamente desarollados, es un mundo en el que constantemente recibimos nuevas noticias de un sistema que mejora al anterior, o que consigue hacer cosas nunca antes vistas. Así que podríamos decir que es menos maduro y tiene muchísimo margen de mejora aún.
\\
Algunos de los frameworks más usados para desarrollar \gls{ia} actualmente son TensorFlow y Keras entre otros, debido a su potencial y su sencillo uso, ya que no tienes que ser un experto para poder usarlas y poder programar tú mismo la \gls{ia}.

\section{Importancia del \glsentrylong{ml}}
El \gls{ml} es un área tremendamente importante en la actualidad. No solo en el campo de los videojuegos como se desarrollará en este caso. También se aplica en distintos campos: tanto en motores de búsqueda, análisis de mercados, detección de enfermedades y muchos más. 
\\
Esto es debido a que estos problemas tienen complejidades suelen ser inasumibles con los algoritmos conocidos, porque se trata de complejidades NP-Hard, es decir, como mínimo, tan complejos como \gls{np}\footnote{Es el conjunto de problemas que pueden ser resueltos en tiempo polinómico por una máquina de Turing no determinista.}. 
\\
El \gls{ml} se encarga de resolver estos problemas, de tal manera que a partir de una pequeña parte de todo el espectro de datos que concierne al problema original \gls{np}, se consigue generalizar una o varias funciones que separan los datos de todo el espectro con el menor error posible.

Por muy avanzada que sea la \gls{ia}, todavía es necesario que la inteligencia de un humano intervenga para desarrollar un nuevo sistema enfocado en un aspecto en concreto. Esto es porque no existe una \gls{ia} capaz de programar un programa complejo, aunque sí se ha demostrado ya que existen \gls{ia} capaces de programar, como es el caso de GPT-3, que usa deep learning, un campo derivado del \gls{ml}.

De manera general, podemos decir que el verdadero potencial del \gls{ml} es la capacidad de analizar datos a la velocidad de un ordenador, y aplicar una lógica aprendida con esos datos para realizar predicciones.
\\
Sin embargo, es una tecnología con la que hay que tener mucho cuidado, debido a que se suele utilizar junto al Big Data, que es la forma en la que llamamos a recoger una gran cantidad de datos para analizarlos, ya que como estos datos no están revisados, estas máquinas inteligentes pueden verse afectadas por sesgos negativos. Como casos importantes tenemos el algoritmo de detección de fotos de Google, que identificaba a algunas personas negras como gorilas, o el bot de Twitter de Microsoft, que aprendió comportamientos racistas y machistas en base a estudiar el comportamiento de esta red. 

\section{Finalidad del proyecto}
Ya que no soy un desarrollador de videojuegos, no pretendo que el resultado final sea tan profesional, a nivel audiovisual, como un juego comercial, porque el objetivo principal es el desarrollo del código. Sin embargo, intentaré que el juego tenga un acabado competente. 

En cuanto a la finalidad del juego es la siguiente: la \gls{ia} es algo extremadamente difícil de comprender, sobretodo si no tienes experiencia programando, por lo tanto cuando una persona juegue por primera vez se tratará de que la persona pueda experimentar con los distintos parámetros de la red neuronal, de esta manera será más accesible para una persona sin conocimientos sobre el tema llegar a comprender el verdadero funcionamiento de una red neuronal y de cómo aprende, al menos de manera general. 
\\
Esto se tratará de conseguir con los diferentes modos, entre los que el jugador podrá entrenarla, ajustar sus parámetros, y ponerla en acción para enfrentarse contra ella manualmente, o poner a dos \gls{ia} distintas a enfrentarse entre ellas.
\\
Una vez conseguido que el jugador comprenda las bases de la \gls{ia}, podría seguir informándose por su cuenta, pues la finalidad del juego no es que la persona que lo juegue se vuelva un experto en el mundo, sino que más bien servirá a nivel introductorio. 

En última instancia, la finalidad del juego es entretener al mismo tiempo que el jugador aprende. Por lo que se intentará que sea un reto entretenido poder enfrentar tu \gls{ia} entrenada contra la de un amigo que proporcione la suya para competir. De esta manera se genera una competencia en la que tienes que esforzarte en aprender qué acciones hacen mejorar la inteligencia de tu bot, con lo que la motivación del jugador por aprender cómo funciona realmente la \gls{ia} crecerá.