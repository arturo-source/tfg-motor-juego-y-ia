%%%%%%%%%%%%%%%%%%%%%%%%%%%%%%%%%%%%%%%%%%%%%%%%%%%%%%%%%%%%%%%%%%%%%%%%
% Plantilla TFG/TFM
% Escuela Politécnica Superior de la Universidad de Alicante
% Realizado por: Jose Manuel Requena Plens
% Contacto: info@jmrplens.com / Telegram:@jmrplens
%%%%%%%%%%%%%%%%%%%%%%%%%%%%%%%%%%%%%%%%%%%%%%%%%%%%%%%%%%%%%%%%%%%%%%%%

\chapter{Introducción}
\label{introduccion}
\todo{Duda sobre el documento, escribo en varios tiempos verbales inconscientemente, también porque el TFG ha sido escrito en distintos momentos, ¿esto tendría que corregirlo en la versión final? De ser así, ¿qué tiempo verbal es el adecuado?}
\textit{IA diseñada para videojuegos} es el primer trabajo de la universidad que ha sido  elegido por mí al 100\%. Esto dice mucho del proyecto porque no he sido obligado a desarrollar una temática, sino que he elegido, en base a mi experiencia en la carrera, lo que es mi parte favorita de la informática. Por lo tanto, lo que vas a encontrar en este \gls{tfg} va a ser lo mejor de mi faceta informática ya que le he dedicado todo el cariño, esfuerzo y calidad que estaba en mi mano aportar.

En la estructura de este proyecto se pueden diferenciar claramente tres partes: el motor del videojuego, el juego en sí, y la \gls{ia} que acompaña al mismo. Y lo que vas a encontrar en este \gls{tfg} no es un juego profesional, ya que no es la parte que más me ha interesado al plantear el proyecto, sino que el centro de la atención es que el jugador sea la persona que aprende a entrenar a una máquina. 
\\
Debido a la difícil barrera de entrada que tiene el mundo de la \gls{ia}, mucha gente que no tiene conocimientos sobre programación es incapaz de entender cómo funciona realmente y se dejan asustar por titulares sensacionalistas de los medios de comunicación. Pero con un juego en el que tienes que entrenar a tu agente, prácticamente cualquier persona podría llegar a comprender, de manera general, qué es realmente una \gls{ia} y no caer en engaños debido al desconocimiento.

Además, como el mundo de los videojuegos me resulta algo tremendamente interesante a nivel del desarrollo de los mismos, pero soy una persona sin experiencia en esto último, me interesaba aprender cómo funciona hasta el más mínimo detalle. Por este motivo he decidido desarrollar yo mismo el motor del videojuego, en lugar de elegir un motor ya existente. Esto habría acortado el tiempo de desarrollo del proyecto, pero no habría conseguido el conocimiento tras desarrollar y analizar cada punto del desarrollo del juego.

Ya que no soy un desarrollador de videojuegos, no pretendo que el resultado final sea tan profesional, a nivel audiovisual, como un juego comercial, porque el objetivo principal es el desarrollo del código. Sin embargo, intentaré que el juego tenga un acabado competente. 

Y en cuanto al documento que estás leyendo, se presentarán los apartados de \textit{Marco teórico}, en el cual comentaré los detalles y los trabajos ya hechos sobre los que se basa mi proyecto, \textit{Objetivos} que será una lista precisa de los objetivos de este, \textit{Metodología} donde cuento cómo he trabajado durante el desarrollo y cómo ha sido el trabajo con mi tutor, \textit{Desarrollo} donde se habla extensamente de principio a fin del desarrollo del \gls{tfg}, y por último \textit{Conclusiones} que, como el propio nombre indica, mostraré las conclusiones a las que he llegado después de haber desarrollado el proyecto.

\section{Objetivos del juego}
El primer objetivo del proyecto es crear un motor, lo suficientemente genérico como para que pueda ser usado por otros videojuegos. Este motor estará separado de las mecánicas del juego como son el comportamiento de las físicas, las colisiones, o el renderizado, ya que todas estas funcionalidades son diferentes en cada juego, dependerá de si queremos hacer un renderizado 3D, 2D, si queremos un comportamiento distinto de las colisiones, etc. Por lo que crearé un motor para el videojuego, con los conocimientos que habré adquirido mediante la formación previa.

El principal objetivo de este proyecto es desarrollar un juego en el que la persona que se decida a jugarlo aprenda los conocimientos básicos sobre \gls{ml}. La tarea del jugador será entrenar a la \gls{ia} de manera que sirva como un oponente capaz de ganar a otros jugadores, y también a otras \gls{ia}.
\\
El usuario podrá tanto entrenar al agente en una arena de juego, como poner al agente a entrenar con los datos extraídos de alguna partida, o los datos extraídos de la arena, e incluso ajustar los parámetros de la \gls{ia} manualmente para experimentar con ellos. También, el jugador podrá enfrentarse contra el agente, y en última instancia, el objetivo final es enfrentar la \gls{ia} de un jugador contra la de otro jugador que ofrezca su \gls{ia} como rival. 
\\
El concepto de la idea es bastante sencillo, crearé el conocido juego Pong, un juego con mecánicas básicas y fácil de entender, y añadiré dificultades al mismo, que puede ser trivial para un humano superarlas, pero algo más difícil para una máquina. La motivación para añadir estas dificultades al juego es que el juego del Pong tiene tan pocos elementos (una pala rival, una pala que eres tú mismo, y la pelota) que hace muy sencillo que una \gls{ia} pueda aprender a jugar contra un humano a un nivel decente. Es tanto así, que en una parte del  desarrollo habré creado solo el Pong, y podré a entrenar a dos perceptrones, y veremos en los resultados si con sólo dos perceptrones es posible que una \gls{ia} aprenda a jugar al Pong.
\\
Posteriormente, cuando haya añadido dificultades al juego, programaré tecnologías más avanzadas que el perceptrón, como puede ser una red neuronal. Y entonces haré pruebas con estas nuevas dificultades añadidas, para ver si realmente el agente consigue aprender a jugar a este remix del juego del Pong.
\\
Y por último, añadiré una interfaz amigable, para que el producto final pueda ser manejado por un usuario sin necesidad de editar el código o ficheros de configuración, sino que pueda hacerlo todo desde la interfaz. Tanto entrenar, como seleccionar los conjuntos de datos para la \gls{ia}, como ajustar parámetros, como elegir la opción de jugar, todo desde la interfaz.

\section{Finalidad del proyecto}
En última instancia, la finalidad del juego es entretener al mismo tiempo que el jugador aprende. Por lo que se intentará que sea un reto entretenido poder enfrentar tu \gls{ia} entrenada contra la de un amigo que proporcione la suya para competir. De esta manera se genera una competencia en la que tienes que esforzarte en aprender qué acciones hacen mejorar la inteligencia de tu bot, con lo que la motivación del jugador por aprender cómo funciona realmente la \gls{ia} crecerá.
\\
Esto se tratará de conseguir con los diferentes modos, entre los que el jugador podrá entrenarla, ajustar sus parámetros, y ponerla en acción para enfrentarse contra ella manualmente, o poner a dos \gls{ia} distintas a enfrentarse entre ellas.
\\
Una vez conseguido que el jugador comprenda las bases de la \gls{ia}, podría seguir informándose por su cuenta, pues la finalidad del juego no es que la persona que lo juegue se vuelva un experto en el mundo, sino que más bien servirá a nivel introductorio. 
